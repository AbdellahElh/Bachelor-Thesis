%%=============================================================================
%% Voorwoord
%%=============================================================================

\chapter*{\IfLanguageName{dutch}{Woord vooraf}{Preface}}%
\label{ch:voorwoord}

%% TODO:
%% Het voorwoord is het enige deel van de bachelorproef waar je vanuit je
%% eigen standpunt (``ik-vorm'') mag schrijven. Je kan hier bv. motiveren
%% waarom jij het onderwerp wil bespreken.
%% Vergeet ook niet te bedanken wie je geholpen/gesteund/... heeft

This thesis was written in the context of my graduation from the Applied Computer Science program at the University College Ghent. This thesis serves as an exploration into the implementation of a facial-recognition-based authentication system for password management.

The entire first semester of the 2024-2025 academic year, I worked on writing and creating the research proposal and implementing a basic authentication system that exclusively utilized facial recognition to evaluate the project's feasibility. The actual implementation of the final prototype, along with writing this thesis, began in the second semester and proved to be a substantial challenge.

I chose this topic based on personal experience with password managers, where I found it a bit annoying to repeatedly enter a master password in the browser while the mobile version of the same provider operated seamlessly with biometric authentication alone. This discrepancy inspired me to explore the integration of facial recognition for more user-friendly and secure access management.

It was reassuring when I discussed the concept with Professor Eric Verbeiren, who encouraged its potential usefulness. He expressed that such a system would benefit individuals, including himself, who struggle with remembering passwords. His feedback and encouragement further motivated me to pursue this topic and explore its practical implications.

I would like to express my gratitude to several individuals who have supported me throughout this journey. First and foremost, I extend my thanks to my 
% co-promoter, Christophe Dutoict, for his guidance, patience, and expertise. His insights and assistance were essential in overcoming technical challenges and refining the project’s implementation, especially during the registration process.

% I am also deeply grateful to my 
supervisor, Andreas De Witte, for his continuous support and constructive feedback throughout my research. His guidance has been helpful in shaping the direction of this thesis and improving its academic rigor.

Lastly, I would like to thank my friend, Jari Vancauwenberghe, for dedicating time to read my thesis and testing the prototype. His feedback and suggestions were greatly helpful in refining both the thesis and the project itself.
