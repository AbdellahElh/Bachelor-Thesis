%%=============================================================================
%% Methodologie
%%=============================================================================

\chapter{\IfLanguageName{dutch}{Methodologie}{Methodology}}%
\label{ch:methodologie}

The development process of this project was structured into distinct stages, each designed to achieve specific objectives, produce clear deliverables, and employ targeted research methods. The completed stages are detailed below.

\subsection{Tool Selection}
The goal of this stage was to select an appropriate facial recognition tool optimized for accessibility, security, and ease of integration. Following an extensive literature review and comparative analysis, \textbf{face-api.js} was chosen as the primary tool, primarily based on the following criteria:
\begin{itemize}
\item \textbf{Privacy:} Face-api.js processes data entirely client-side, thereby eliminating the need to transmit sensitive biometric information externally and enhancing user privacy.
\item \textbf{Compatibility:} The library integrates smoothly with modern JavaScript frameworks, facilitating straightforward development and reducing complexity.
\item \textbf{Real-time Performance:} Its lightweight and modular architecture allows efficient real-time operation, making it particularly suitable for accessibility-focused web applications.
\end{itemize}

Alternatives such as \textbf{OpenCV (Java)} and \textbf{face\_recognition (Python)} were considered and benchmarked, but ultimately face-api.js was selected due to its superior balance of performance, ease of integration, and suitability for client-side web deployment.

\subsection{Prototype Development}
The primary purpose of this phase is the development of a fully functional web-based prototype password manager designed specifically to address usability and accessibility concerns for individuals with cognitive and motor disabilities. The prototype is being developed using TypeScript for enhanced maintainability and robustness. Key deliverables completed in this stage include:
\begin{itemize}
\item \textbf{Face Authentication Module:} Users authenticate effortlessly through facial recognition, removing the complexity associated with traditional textual passwords.
\item \textbf{Credential Management System:} Implementation of secure methods for storing and retrieving user credentials.
\item \textbf{Password Generation Tool:} An integrated mechanism that automatically generates robust, unique passwords.
\item \textbf{Accessible User Interface (UI):} Designed according to Web Content Accessibility Guidelines (WCAG), ensuring ease of use, particularly for users facing cognitive or motor challenges.
\end{itemize}

The development followed an iterative Agile methodology, allowing regular refinements based on intermediate testing and feedback from preliminary user evaluations.

\subsection{Database Selection}
Selecting an efficient and secure database solution to manage user credentials was a crucial component of the project. After evaluating various database options against performance, security, scalability, and ease of integration criteria, the final selection included:
\begin{itemize}
\item \textbf{SQLite:} Chosen for the initial prototype due to its simplicity, lightweight nature, and suitability for local, client-side storage.
\item \textbf{MongoDB:} Assessed for its scalability, flexibility, and potential to handle more complex data structures and larger datasets in future expanded implementations.
\end{itemize}

The database solutions were validated through performance tests and security assessments conducted during the prototype development phase.


\subsection{Future Expansion and Scalability}
With the successful demonstration of the web-based prototype’s advantages in accessibility and usability, the groundwork has been established for future expansions onto other platforms, such as desktop and mobile applications. Planned next steps include:
\begin{itemize}
\item \textbf{Technology Evaluation:} Comparative analysis of cross-platform frameworks (e.g., Electron, React Native) focusing on compatibility, ease of code migration, and performance considerations.
\item \textbf{Scalability Planning:} Developing strategies for accommodating larger user bases, maintaining system performance, and ensuring consistent security standards across different platforms.
\end{itemize}

%% TODO: In dit hoofstuk geef je een korte toelichting over hoe je te werk bent
%% gegaan. Verdeel je onderzoek in grote fasen, en licht in elke fase toe wat
%% de doelstelling was, welke deliverables daar uit gekomen zijn, en welke
%% onderzoeksmethoden je daarbij toegepast hebt. Verantwoord waarom je
%% op deze manier te werk gegaan bent.
%% 
%% Voorbeelden van zulke fasen zijn: literatuurstudie, opstellen van een
%% requirements-analyse, opstellen long-list (bij vergelijkende studie),
%% selectie van geschikte tools (bij vergelijkende studie, "short-list"),
%% opzetten testopstelling/PoC, uitvoeren testen en verzamelen
%% van resultaten, analyse van resultaten, ...
%%
%% !!!!! LET OP !!!!!
%%
%% Het is uitdrukkelijk NIET de bedoeling dat je het grootste deel van de corpus
%% van je bachelorproef in dit hoofstuk verwerkt! Dit hoofdstuk is eerder een
%% kort overzicht van je plan van aanpak.
%%
%% Maak voor elke fase (behalve het literatuuronderzoek) een NIEUW HOOFDSTUK aan
%% en geef het een gepaste titel.

% \lipsum[21-25]

