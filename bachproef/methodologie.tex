%%=============================================================================
%% Methodologie
%%=============================================================================

\chapter{\IfLanguageName{dutch}{Methodologie}{Methodology}}%
\label{ch:methodologie}

The development process for this project is structured into distinct stages, each pursuing specific objectives, generating clear deliverables, and applying targeted research methods. The stages appear below.

\section{Tool Selection}
This stage selects a facial-recognition tool that balances accessibility, security and ease of integration.  
Following an extensive literature review and comparative analysis, the project adopts \textbf{face-api.js} based on:

\begin{itemize}
  \item \textbf{Privacy:} Face-api.js processes all data on the client, eliminating external transmission of biometric information and strengthening user privacy.
  \item \textbf{Compatibility:} The library integrates smoothly with modern 
  Java\-Script frameworks, streamlining development and reducing complexity.
  \item \textbf{Real-time performance:} Its lightweight, modular architecture enables efficient real-time operation—an essential feature for accessibility-centred web applications.
\end{itemize}

Alternatives such as \textbf{OpenCV (Java)} and \textbf{face\_recognition (Python)} are considered and benchmarked, yet face-api.js ultimately offers the best balance of performance, integration effort and suitability for client-side deployment.

\section{Prototype Development}
This phase describes the development of a fully functional, web-based password manager that addresses usability and accessibility barriers for individuals with cognitive and motor disabilities. The prototype relies on TypeScript to maximise maintainability and robustness. Key deliverables in this stage include:

\begin{itemize}
  \item \textbf{Face-authentication module:} Users authenticate through facial recognition, removing the burden of remembering complex textual passwords.
  \item \textbf{Credential-management system:} Secure mechanisms store and retrieve user credentials.
  \item \textbf{Password-generation tool:} An integrated utility produces robust, unique passwords automatically.
  \item \textbf{Accessible user interface (UI):} The design follows Web Content Accessibility Guidelines (WCAG) to ensure usability—especially for users facing cognitive or motor challenges.
\end{itemize}

Development follows an iterative Agile methodology, enabling continuous refinement through intermediate testing and preliminary user feedback.

\section{Database Selection}
Choosing an efficient, secure database to manage user credentials forms a crucial component of the project. After evaluating multiple options against performance, security, scalability and integration criteria, the project currently employs:

\begin{itemize}
  \item \textbf{SQLite:} Adopted for the initial prototype because of its simplicity, lightweight footprint and suitability for local client-side storage.
  \item \textbf{PostgreSQL:} Evaluated for future scenarios that demand greater scalability, flexible data models and larger datasets.
\end{itemize}

Both database solutions undergo performance tests and security assessments during prototype development.

\section{Future Expansion and Scalability}
If the browser-based prototype proves successful, the next plan is to
package it as an \textbf{offline-first Chrome extension}.  
SQLite's embedded design keeps the entire password vault in a single file
bundled with the extension, so no external server is required—precisely why
SQLite was chosen.  This direction will let users install the password manager
instantly from the Chrome Web Store and keep their credentials local while
still benefiting from auto-fill and seamless updates.

Planned next steps include:
\begin{itemize}
  \item Adapting the current React/TypeScript codebase to Chrome Extension
        APIs (manifest v3) for secure content-script injection and
        background tasks.
  \item Implementing permission-scoped access to web pages for auto-fill
        while preserving privacy.
  \item Testing storage limits and performance of the SQLite WASM build in
        Chrome to ensure smooth operation on low-end devices.
  \item Exploring optional cloud backup and multi-device sync as opt-in
        features, keeping the default experience fully offline.
\end{itemize}


%% TODO: In dit hoofstuk geef je een korte toelichting over hoe je te werk bent
%% gegaan. Verdeel je onderzoek in grote fasen, en licht in elke fase toe wat
%% de doelstelling was, welke deliverables daar uit gekomen zijn, en welke
%% onderzoeksmethoden je daarbij toegepast hebt. Verantwoord waarom je
%% op deze manier te werk gegaan bent.
%% 
%% Voorbeelden van zulke fasen zijn: literatuurstudie, opstellen van een
%% requirements-analyse, opstellen long-list (bij vergelijkende studie),
%% selectie van geschikte tools (bij vergelijkende studie, "short-list"),
%% opzetten testopstelling/PoC, uitvoeren testen en verzamelen
%% van resultaten, analyse van resultaten, ...
%%
%% !!!!! LET OP !!!!!
%%
%% Het is uitdrukkelijk NIET de bedoeling dat je het grootste deel van de corpus
%% van je bachelorproef in dit hoofstuk verwerkt! Dit hoofdstuk is eerder een
%% kort overzicht van je plan van aanpak.
%%
%% Maak voor elke fase (behalve het literatuuronderzoek) een NIEUW HOOFDSTUK aan
%% en geef het een gepaste titel.

% \lipsum[21-25]

