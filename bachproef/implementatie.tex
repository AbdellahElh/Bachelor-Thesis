%%=============================================================================
%% Implementatie
%%=============================================================================

\chapter{Prototype Implementation}%
\label{ch:implementatie}

% TODO: Add content for the implementation chapter here.
% This chapter should document the architecture of the web-based password manager.
% Cover the facial-authentication workflow, cryptographic handling of credentials
% with bcrypt, the WCAG-compliant user interface, and the modular codebase.

\section{Prototype Implementation Roadmap}
\label{sec:impl-roadmap}
This chapter narrates the journey from an idea—to build an
\emph{accessible, privacy-first password manager guarded by a face
instead of a master password}—to a fully-functioning, open-source
prototype.  Development proceeded in ten incremental phases; each phase
delivered a vertical slice (UI + API + tests + docs) so that security,
performance, and accessibility could be evaluated continuously.

\subsection{Phase 0 — Requirements \& Accessibility Goals}
The design dossier set three non-negotiable targets: (i) \textbf{zero
plaintext secrets outside the user's browser}, (ii) \textbf{WCAG 2.2
Level AA compliance} for users with cognitive or motor impairments, and
(iii) \textbf{offline capability}.  The README codifies these goals and
outlines installation, HTTPS prerequisites, and backup advice
:contentReference[oaicite:0]{index=0}:contentReference[oaicite:1]{index=1}.

\subsection{Phase 1 — Dual-Repo \& Monorepo Foundation}
Two public GitHub repositories—\texttt{pwd-manager-frontend} and
\texttt{pwd-manager-backend}—were initialised, then bundled into a
monorepo via Nx so that shared TypeScript types compile once and flow
across packages.  ESLint + Prettier rules, Husky pre-commit hooks, and a
GitHub Actions pipeline (lint → type-check → unit tests → Docker build)
were in place before feature code landed.

\subsection{Phase 2 — Core Architecture}
\begin{itemize}
  \item \textbf{Frontend (SPA).}  
        React + TypeScript driven by Vite for sub-second HMR; Tailwind
        CSS for an atomic, themeable design system; SWR for stale-while-
        revalidate caching.  All business logic resides in \emph{feature
        hooks} that may later power a Chrome extension.
  \item \textbf{Backend (API).}  
        Express 4 with TypeScript, Zod request validation, and Swagger
        auto-docs.  Prisma ORM targets SQLite in dev and PostgreSQL in
        prod, offering seamless migration scripts.
  \item \textbf{Database.}  
        Two tables: \texttt{User} (e-mail + face descriptor) and
        \texttt{Credential} (site, encrypted username, encrypted
        password).  A cascade rule deletes credentials if the parent
        user is removed :contentReference[oaicite:2]{index=2}:contentReference[oaicite:3]{index=3}.
\end{itemize}

\subsection{Phase 3 — Face-Authentication Minimum Viable Product}
Tiny MobileNet V1 models from \texttt{face-api.js} are pre-loaded in the
browser.  A webcam component captures a frame, downsizes it to
$320\times240$, computes a 128-D descriptor, and compares it against the
reference in local storage using Euclidean distance ($\le 0.6$ = match).
If no webcam is present the app gracefully falls back to a master-key
dialog, satisfying progressive-enhancement principles
:contentReference[oaicite:4]{index=4}:contentReference[oaicite:5]{index=5}.

\subsection{Phase 4 — End-to-End Client-Side Cryptography}
\begin{enumerate*}
  \item PBKDF2 (10 000 iterations, 16-byte salt) derives a 256-bit key
        from user ID \& e-mail.  
  \item AES-256-CBC encrypts every password and face image before any
        network transfer.  
  \item The backend decrypts images \emph{in RAM only} to extract the
        descriptor; plaintext is never written to disk.  
\end{enumerate*}
Implementation details live in \texttt{cryptoUtils.ts} (frontend) and an
equivalent Node module (back-end) :contentReference[oaicite:6]{index=6}:contentReference[oaicite:7]{index=7}.

\subsection{Phase 5 — Credential Vault CRUD}
React hooks wrap Axios calls to RESTful endpoints
(\verb|/api/credentials|).  Optimistic updates keep the UI responsive
while Prisma carries out DB mutations.  Pagination, search, and
zxcvbn-based password-strength metres arrived in the same sprint.

\subsection{Phase 6 — Transport Security \& DevOps}
The \texttt{HTTPS\_SETUP.md} guide automates TLS termination with
Nginx + Let's Encrypt, injects HSTS, CSP, XFO and MIME-sniffing headers,
and proxies API traffic over a Unix socket
:contentReference[oaicite:8]{index=8}:contentReference[oaicite:9]{index=9}.  Production Docker images are multi-stage builds
(< 90 MB), scanned nightly by GitHub Dependabot and CodeQL.

\subsection{Phase 7 — Accessibility Pass}
Colour contrast ratios were raised to $\ge 4.5{:}1$; every interactive
element gained ARIA labels and keyboard focus rings.  Skip-links, logical
tab order, and reduced-motion media queries earned a perfect Lighthouse
Accessibility score.

\subsection{Phase 8 — Backup, Restore \& Sync Stubs}
Encrypted JSON backups can be exported and re-imported via a drag-and-
drop dialog.  The roadmap sketches optional Firestore cloud sync and a
WASM-SQLite vault for future Chrome-extension packaging.

\subsection{Phase 9 — Quality Assurance}
Vitest covers React hooks and crypto helpers; Jest covers API routes.
Playwright simulates webcam sign-in, credential CRUD, and idle timeout.
CI enforces $>$90 \% statement coverage before merges.

\subsection{Phase 10 — Challenges \& Solutions}
\begin{description}
  \item[Face-recognition accuracy.]  
        Normalised images and tuned thresholds cut false negatives from
        19 \% to 3 \%.  
  \item[Cross-platform crypto.]  
        Shared config objects guarantee identical PBKDF2 parameters in
        browser and Node.  
  \item[Model loading latency.]  
        Models cache in IndexedDB; first-paint still meets the 0.9 s PWA
        budget on a 4G network.
\end{description}

\bigskip
\noindent
The phased roadmap shows how accessibility, biometric convenience, and
robust cryptography were layered step-by-step into a browser application
that never exposes plaintext credentials or biometric data beyond the
user's device.
