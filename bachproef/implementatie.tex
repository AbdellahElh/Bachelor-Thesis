
%% Implementatie
%%=============================================================================

\chapter{Prototype Implementation}%
\label{ch:implementatie}

% TODO: Add content for the implementation chapter here.
% This chapter should document the architecture of the web-based password manager.
% Cover the facial-authentication workflow, cryptographic handling of credentials
% with bcrypt, the WCAG-compliant user interface, and the modular codebase.

\section{Prototype Implementation}

\subsection{Architecture overview}
The password manager is implemented as a modern client-server web application.  
\begin{itemize}
  \item \textbf{Frontend} - a React single-page application (SPA) written in TypeScript delivers the user interface.  
  \item \textbf{Backend} - a Node.js\,/\,Express service, also in TypeScript, handles data persistence, face-recognition logic and JSON Web Token (JWT) issuance.  
  \item \textbf{Database} - user and credential data are stored in an SQLite file accessed through the Prisma ORM; a PostgreSQL target is available for future scaling.  
\end{itemize}

\subsection{Technology stack}
\paragraph{Frontend}
React {+} TypeScript, React Context API for state, Tailwind CSS components, Axios for REST calls, \texttt{crypto-js} for client-side AES-256 encryption, and \texttt{face-api.js} (TensorFlow.js) for in-browser face detection.

\paragraph{Backend}
Node.js 18, Express 4, TypeScript, Prisma, \texttt{face-api.js} running under \texttt{node-canvas} for server-side inference, JWT for session tokens, and the Node \texttt{crypto} module for cryptography.

\subsection{Face-authentication workflow}
\begin{enumerate}
  \item \textbf{Registration} - the browser captures a webcam frame, encrypts it with AES-256 under a user-specific key, and posts it to the server. After decryption, \texttt{face-api.js} extracts a 128-D descriptor that is stored with the user record.
  \item \textbf{Login} - a fresh frame is captured, encrypted and processed in the same way; its descriptor is compared to the stored vector with Euclidean distance~$\le0.6$. Our target threshold (FAR $\leq 0.1\%$, FRR $\leq 10\%$) is grounded in the industry figures summarised in Section~\ref{sec:biometric-benchmarks}.
  \item \textbf{Optimisation} - images are down-scaled (320\,$\times$\,240) and processed with TinyFaceDetector / MobileNet V1 models that are pre-loaded once at server start.
\end{enumerate}

\subsection{Secure credential vault}
\begin{itemize}
  \item Keys are derived client-side with PBKDF2 (10\,000 iterations, 16-byte salt).  
  \item All usernames and passwords are AES-256-CBC encrypted in the browser before transmission; only ciphertext is stored.  
  \item During retrieval the encrypted blobs are returned unchanged and decrypted locally, so plaintext secrets never leave the user's device.  
\end{itemize}

\subsection{Data protection and GDPR compliance}
From a privacy perspective, facial data is classified as a special category of personal data under Article 9 of the EU General Data Protection Regulation (GDPR) \autocite{GDPR2016}, which means its processing is generally prohibited unless specific conditions (such as the data subject's explicit consent) are met. In recognition of these strict requirements, the prototype is designed with data minimization and user consent in mind. All facial images captured for authentication are immediately encrypted on the client side and never stored in plaintext or transmitted off the user's device, ensuring that no raw biometric data is persistently retained.

\subsection{Implementation roadmap}
Development progressed through the following incremental phases:

\begin{enumerate}[label=\textbf{Phase~\arabic*:}, leftmargin=2.8em]

  \item \emph{Foundation \& Project Setup}  
        Create two repositories (\texttt{pwd-manager-frontend}, \texttt{pwd-manager-backend}) with a common Nx
        workspace, React\,+\,TypeScript front-end and Node.js\,+\,Express back-end.
        Initialise Prisma with an SQLite schema that removes the traditional password field
        and adds 128-D face-descriptor columns.

  \item \emph{Core Authentication Pipeline}  
        Load face-api.js models at server start, expose registration and login endpoints, and
        integrate in-browser face matching (Tiny MobileNet V1) with React context for
        session state.

  \item \emph{Client-Side Cryptography}  
        Implement zero-knowledge design: PBKDF2-derived keys and AES-256-CBC encryption
        in the browser, mirrored by decryption helpers on the back-end
        (see Section~\ref{sec:crypto} for the rationale behind AES and PBKDF2).

  \item \emph{Credential Vault CRUD}  
        Add REST routes for creating, reading, updating and deleting encrypted credentials.
        Provide optimistic UI updates, a password-strength meter, and toast notifications.

  \item \emph{Security Hardening}  
        Introduce HTTPS enforcement, secure headers, JWT authentication middleware,
        unique user-specific encryption keys, and detailed error logging for crypto
        operations.  Update \texttt{.env.example} and validation checks for all secrets.

  \item \emph{UI/UX \& Accessibility Pass}  
        Refactor login and registration forms, apply Tailwind design tokens, fix colour-contrast,
        tab order and ARIA labels to meet WCAG 2.2 AA.

  \item \emph{Data Portability \& Sync Stubs}  
        Implement backup and restore via encrypted JSON, add preliminary endpoints for
        future cross-device synchronisation, and optimise user-image storage.

  \item \emph{Production Readiness}  
        Multi-stage Dockerfiles, Nginx reverse proxy with HSTS and CSP, Vitest + Jest unit
        tests, Playwright end-to-end tests, and a CI pipeline targeting \textgreater90 \% coverage.
        Final README and code clean-up.

\end{enumerate}



\subsection{Database schema}
The \texttt{User} table stores the e-mail address and the 128-dimensional face descriptor; the \texttt{Credential} table stores website metadata together with encrypted username and password, and cascades on user deletion.


\subsection{Encryption and Decryption}\label{subsec:encryption}

All sensitive fields including passwords and webcam images are protected end-to-end with AES-256 encryption.
Keys are never transmitted in plaintext; instead they are derived client-side with PBKDF2 and reproduced server-side only when needed for decryption.

\subsubsection{Password encryption workflow}

\paragraph{Key derivation (PBKDF2).}
A 256-bit key is derived in the browser from a user-specific base key and a 16-byte salt, using 10000 iterations.

\begin{lstlisting}[language=TypeScript, caption={Key derivation function using PBKDF2}, label={lst:strengthen-key}]
export const strengthenKey = (baseKey: string): string => {
  return PBKDF2(baseKey, SALT, {
    keySize: KEY_SIZE / 32,   // 256 bits
    iterations: ITERATIONS,   // 10,000
  }).toString();
};
\end{lstlisting}

\paragraph{AES-256 encryption and decryption.}
The code below demonstrates the encryption and decryption functions used in the client. The encrypt function strengthens the provided key and uses AES to encrypt the value, while the decrypt function reverses this process with error handling.

\begin{lstlisting}[language=TypeScript, caption={AES-256 encryption and decryption functions}, label={lst:encrypt-decrypt}]
export const encrypt = (value: string, secretKey: string): string => {
  if (!value) return "";
  const strengthenedKey = strengthenKey(secretKey);
  return AES.encrypt(value, strengthenedKey).toString();
};

export const decrypt = (
  encryptedValue: string,
  secretKey: string
): string => {
  if (!encryptedValue) return "";
  try {
    const strengthenedKey = strengthenKey(secretKey);
    const bytes = AES.decrypt(encryptedValue, strengthenedKey);
    return bytes.toString(UTF8);
  } catch (error) {
    console.error("Failed to decrypt value:", error);
    return "";
  }
};
\end{lstlisting}

Each user obtains a unique secret based on their ID, e-mail and an application secret, as shown in the following function.

\begin{lstlisting}[language=TypeScript, caption={User-specific encryption key generation}, label={lst:user-key}]
export const getUserEncryptionKey = (
  userId: number,
  userEmail: string
): string => {
  const appSecretKey = import.meta.env.VITE_SECRET_KEY || "app-default-key";
  return `pwd-manager-${userId}-${userEmail}-${appSecretKey}`;
};
\end{lstlisting}

Before storage, credential fields are encrypted using the function below, which processes each sensitive field individually.

\begin{lstlisting}[language=TypeScript, caption={Encryption of credential fields before storage}, label={lst:encrypt-credential}]
const encryptCredential = (
  credential: CredentialEntry,
  encryptionKey: string
): EncryptedCredential => {
  return {
    ...credential,
    username: encrypt(credential.username, encryptionKey),
    password: encrypt(credential.password, encryptionKey),
    notes: credential.notes
      ? encrypt(credential.notes, encryptionKey)
      : undefined,
  };
};
\end{lstlisting}

\subsubsection{Face-image encryption workflow}

\paragraph{Client-side processing.}
A captured webcam frame is converted to Base64 and then encrypted with the same AES/PBKDF2 scheme, as demonstrated in the code below.

\begin{lstlisting}[language=TypeScript, caption={Client-side webcam image processing and encryption}, label={lst:image-processing}]
export const blobToBase64 = (blob: Blob): Promise<string> => {
  return new Promise((resolve, reject) => {
    const reader = new FileReader();
    reader.onloadend = () => {
      if (typeof reader.result === "string") {
        const base64 = reader.result.split(",")[1]; // strip data URL prefix
        resolve(base64);
      } else {
        reject(new Error("Failed to convert blob to base64"));
      }
    };
    reader.readAsDataURL(blob);
  });
};

export const encryptImage = async (
  imageBlob: Blob,
  encryptionKey: string
): Promise<{ encryptedData: string; contentType: string }> => {
  const base64Data = await blobToBase64(imageBlob);
  const encryptedData = encrypt(base64Data, encryptionKey);
  return { encryptedData, contentType: imageBlob.type };
};
\end{lstlisting}

Encrypted payloads are packaged into a `FormData` object for upload using the following utility function.

\begin{lstlisting}[language=TypeScript, caption={Creation of FormData for encrypted image upload}, label={lst:form-data}]
export const createEncryptedImageFormData = async (
  imageBlob: Blob,
  encryptionKey: string,
  fieldName: string = "image",
  additionalData: Record<string, string> = {}
): Promise<FormData> => {
  const { encryptedData, contentType } =
    await encryptImage(imageBlob, encryptionKey);

  const formData = new FormData();
  formData.append(`${fieldName}Encrypted`, encryptedData);
  formData.append(`${fieldName}ContentType`, contentType);
  formData.append(`${fieldName}IsEncrypted`, "true");
  Object.entries(additionalData).forEach(([k, v]) =>
    formData.append(k, v)
  );
  return formData;
};
\end{lstlisting}

\paragraph{Server-side decryption.}
The backend reproduces the key, then performs AES-256-CBC decryption as demonstrated in the server-side functions below.

\begin{lstlisting}[language=TypeScript, caption={Server-side decryption of encrypted data from the frontend}, label={lst:server-decryption}]
export const decryptFromFrontend = (
  cipherText: string,
  password: string
): Buffer => {
  const parts = cipherText.split(":");        // salt:iv:data
  const salt = Buffer.from(parts[0], "hex");
  const iv   = Buffer.from(parts[1], "hex");
  const encryptedData = Buffer.from(parts[2], "base64");

  const key = deriveKey(password, salt);      // PBKDF2
  const decipher = crypto.createDecipheriv(
    CRYPTO_CONFIG.algorithm, key, iv
  );
  return Buffer.concat([decipher.update(encryptedData), decipher.final()]);
};

export const decryptUserImage = (
  encryptedBase64: string,
  userId: number,
  userEmail: string
): Buffer => {
  const encryptionKey = getUserEncryptionKey(userId, userEmail);
  return decryptFromFrontend(encryptedBase64, encryptionKey);
};
\end{lstlisting}

\subsubsection{Security guarantees}

\begin{itemize}
  \item \textbf{End-to-end encryption}: passwords and images are encrypted in the browser; only ciphertext and face descriptors are stored.  
  \item \textbf{Per-user keys}: each key binds the user's ID and e-mail to an application secret, preventing key reuse across accounts.  
  \item \textbf{Key strengthening}: PBKDF2 with 10 000 iterations thwarts brute-force attacks on weak secrets.  
  \item \textbf{Compatible crypto}: frontend (\texttt{crypto-js}) and backend (\texttt{node:crypto}) use identical AES-256-CBC parameters, ensuring seamless decryption.  
\end{itemize}

