%%=============================================================================
%% Implementatie
%%=============================================================================

\chapter{Prototype Implementation}%
\label{ch:implementatie}

% TODO: Add content for the implementation chapter here.
% This chapter should document the architecture of the web-based password manager.
% Cover the facial-authentication workflow, cryptographic handling of credentials
% with bcrypt, the WCAG-compliant user interface, and the modular codebase.


\section{Project Genesis and Development Roadmap}
The browser-based password manager with face authentication was engineered
from an empty GitHub workspace into a polished prototype through ten
well-defined phases.  Each phase delivered a vertical slice—UI, API, tests,
and documentation—so that security and accessibility could be evaluated
continuously.  Figure~\ref{fig:roadmap} visualises the flow; the narrative
below expands on the underlying engineering decisions, tools, and lessons
learned.

\begin{enumerate}[leftmargin=*,label=\textbf{Phase \arabic*}]
% -------------------------------------------------------------------
\item \textbf{Concept \& Requirements Draft.}  
      Before a single line of code, functional goals (local-first storage,
      face-based login, end-to-end encryption) and non-functional goals
      (WCAG 2.2 compliance, OWASP alignment, offline capability) were captured
      in a living design document.  The README was seeded with a high-level
      feature list and installation notes, establishing the project as
      browser-only and privacy-first \autocite{README2025}. :contentReference[oaicite:0]{index=0}:contentReference[oaicite:1]{index=1}
% -------------------------------------------------------------------
\item \textbf{Repository \& Toolchain Bootstrap.}  
      Two public repositories were created—\texttt{pwd-manager-frontend} and
      \texttt{pwd-manager-backend}.  Shared lint rules (ESLint + Prettier),
      Husky git hooks, and GitHub Actions workflows were added on day 1 to
      gate every push with \emph{lint $\to$ type-check $\to$ test $\to$
      build}.  Dependabot auto-PRs keep NPM dependencies patched, and
      CodeQL scans run nightly to detect security smells.
% -------------------------------------------------------------------
\item \textbf{Frontend Scaffolding.}  
      A Vite-generated React + TypeScript SPA provided instant HMR and
      tree-shaking.  Tailwind CSS offered a utility-first design system,
      while Vitest + React Testing Library constituted the unit-test bed.  The
      inaugural screen—a WCAG-compliant “Hello Dashboard”—verified the build
      pipeline and Lighthouse scores.  State was managed with SWR, chosen for
      its stale-while-revalidate caching and tiny footprint.
% -------------------------------------------------------------------
\item \textbf{Backend Scaffolding.}  
      An Express server (TypeScript) was initialised with Zod for request
      validation and Swagger for live API docs.  Prisma ORM targeted SQLite
      in development and PostgreSQL in production, allowing instant DB spins
      via \texttt{npx prisma migrate dev}.  Jest tests covered health,
      auth-token issuance, and placeholder CRUD endpoints.
% -------------------------------------------------------------------
\item \textbf{Biometric MVP.}  
      The SPA integrated \texttt{face-api.js}; Tiny MobileNet V1 models were
      stored under \texttt{/public/models}.  A webcam component captured live
      frames, extracted 128-D descriptors, and compared them to a single
      reference, unlocking the vault UI on a match threshold of 0.6.  Model
      weights lazy-load in parallel with the splash screen, hiding latency
      from the user.  All processing occurs client-side, so raw images never
      traverse the network \autocite{SECURITYGuide}. :contentReference[oaicite:2]{index=2}:contentReference[oaicite:3]{index=3}
% -------------------------------------------------------------------
\item \textbf{Credential Vault CRUD.}  
      React hooks abstracted credential operations; optimistic updates kept
      the UI snappy while REST calls executed.  Backend routes
      (\texttt{GET/POST /api/credentials} and friends) used Prisma clients
      with index-based pagination.  Database constraints enforced foreign-key
      integrity between \texttt{User} and \texttt{Credential}.
% -------------------------------------------------------------------
\item \textbf{End-to-End Client-Side Encryption.}  
      A dedicated \texttt{cryptoUtils.ts} wrapped \texttt{crypto-js}.  PBKDF2
      (10 000 iterations, 16-byte salt) produced a 256-bit key from user
      metadata; AES-256-CBC encrypted passwords, notes, and face images
      before any network transfer.  Mirror helpers on the server decrypted
      images \emph{in-memory only} to compute descriptors, then zeroised
      buffers.  No plaintext ever rested on disk or travelled the wire
      \autocite{SECURITYGuide}. :contentReference[oaicite:4]{index=4}:contentReference[oaicite:5]{index=5}
% -------------------------------------------------------------------
\item \textbf{Secure Transport \& Deployment.}  
      An Nginx reverse proxy, scripted in \texttt{docs/HTTPS\_SETUP.md},
      terminates TLS with Let's Encrypt certificates, injects HSTS, CSP,
      X-Frame-Options, and X-Content-Type-Options headers, then forwards API
      traffic to the Node back-end over a Unix socket \autocite{HTTPSGuide}. :contentReference[oaicite:6]{index=6}:contentReference[oaicite:7]{index=7}  
      A fallback Node HTTPS server is documented for bare-metal deployments.
% -------------------------------------------------------------------
\item \textbf{Accessibility \& UX Sprints.}  
      WCAG 2.2 techniques—skip-links, ARIA labels, keyboard traps
      elimination—were layered on top.  A zxcvbn-based strength meter and
      responsive dark/light themes improved usability.  All interactive
      elements meet a 4.5:1 contrast ratio, validated by the WAVE browser
      extension.
% -------------------------------------------------------------------
\item \textbf{Quality Assurance \& Release Candidate.}  
      Playwright end-to-end scripts simulated webcam sign-in, credential
      export/import, and session timeout.  CI pipelines demanded 90\%
      statement coverage; Istanbul reports gate-kept merges.  A PWA
      manifest, service-worker precaching, and a desktop “install” prompt
      completed the release candidate, which scores 100/100/100/100 on
      Lighthouse (Performance, Accessibility, Best Practices, SEO).
\end{enumerate}

This phased roadmap illustrates how the application matured from design
sketches into a secure, accessible, and offline-capable vault that never
exposes plaintext credentials or biometric data beyond the user's browser.


