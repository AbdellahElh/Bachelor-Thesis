%%=============================================================================
%% Samenvatting
%%=============================================================================

% TODO: De "abstract" of samenvatting is een kernachtige (~ 1 blz. voor een
% thesis) synthese van het document.
%
% Een goede abstract biedt een kernachtig antwoord op volgende vragen:
%
% 1. Waarover gaat de bachelorproef?
% 2. Waarom heb je er over geschreven?
% 3. Hoe heb je het onderzoek uitgevoerd?
% 4. Wat waren de resultaten? Wat blijkt uit je onderzoek?
% 5. Wat betekenen je resultaten? Wat is de relevantie voor het werkveld?
%
% Daarom bestaat een abstract uit volgende componenten:
%
% - inleiding + kaderen thema
% - probleemstelling
% - (centrale) onderzoeksvraag
% - onderzoeksdoelstelling
% - methodologie
% - resultaten (beperk tot de belangrijkste, relevant voor de onderzoeksvraag)
% - conclusies, aanbevelingen, beperkingen
%
% LET OP! Een samenvatting is GEEN voorwoord!

%%---------- Nederlandse samenvatting -----------------------------------------
%
% TODO: Als je je bachelorproef in het Engels schrijft, moet je eerst een
% Nederlandse samenvatting invoegen. Haal daarvoor onderstaande code uit
% commentaar.
% Wie zijn bachelorproef in het Nederlands schrijft, kan dit negeren, de inhoud
% wordt niet in het document ingevoegd.

\IfLanguageName{english}{%
\selectlanguage{dutch}
\chapter*{Samenvatting}
De toenemende afhankelijkheid van online diensten heeft het belang van veilig wachtwoordbeheer verder vergroot. Traditionele authenticatiemethoden, zoals het onthouden en intypen van complexe wachtwoorden, vormen vooral voor mensen met cognitieve of motorische beperkingen een aanzienlijke barrière. Dit onderzoek verkent daarom het gebruik van gezichtsherkenning als toegankelijk authenticatiemechanisme binnen een wachtwoordmanager.

De centrale onderzoeksvraag luidt: \emph{hoe kan gezichtsherkenning de toegankelijkheid van wachtwoordbeheer verbeteren voor mensen met cognitieve en motorische beperkingen}? We analyseerden diverse gezichtsherkennings-API’s in meerdere programmeertalen en kozen uiteindelijk een hybride architectuur: de browser voert de face-detectie uit, versleutelt het beeld met AES-256 en stuurt het naar de server, waar de volledige biometrische verwerking (landmarks, descriptorextractie en matching) plaatsvindt. Zowel de 128-dimensionale descriptors als de versleutelde selfies worden veilig opgeslagen in een SQLite-database via Prisma ORM.

Op basis van deze keuze is een webgebaseerd prototype ontwikkeld. Indien dit positieve resultaten oplevert, volgt een browserextensie als vervolgstap om schaalbaarheid en effectiviteit in een realistische context te toetsen. Deze iteratieve aanpak maakt het mogelijk technologieën stapsgewijs en onderbouwd te evalueren voordat tot volledige implementatie wordt overgegaan.

Verwachte uitkomsten zijn een aantoonbare daling in cognitieve en fysieke belasting ten opzichte van traditionele wachtwoordmanagers en een hogere gebruikerstevredenheid, terwijl de hybride opzet sterke beveiliging biedt door serverside verificatie en volledige end-to-end-versleuteling.

\selectlanguage{english}
}{}

%%---------- Samenvatting -----------------------------------------------------
% De samenvatting in de hoofdtaal van het document

\chapter*{\IfLanguageName{dutch}{Samenvatting}{Abstract}}

The increasing reliance on online services has elevated the importance of password management in digital security. Traditional authentication methods, such as typing and remembering complex passwords, often pose significant challenges for individuals with cognitive or motor disabilities, including difficulties with memory recall or manual input. These limitations highlight the urgent need for accessible authentication solutions that improve both user satisfaction and usability.

This thesis explores the potential of integrating facial recognition technology as an accessible authentication mechanism within password management systems, specifically targeting individuals with cognitive and motor impairments.

The central research question is: \emph{How can face recognition technology improve accessibility in password management for individuals with cognitive and motor disabilities?} The research includes a comparative analysis of various facial recognition APIs across multiple programming languages to determine the most appropriate technology stack for this use case. Based on these findings, a web-based prototype of a password manager was developed using the selected technologies. The prototype follows a hybrid architecture in which the browser executes face detection and client-side encryption of captured images, while the server performs descriptor extraction, matching, and secure storage.

If the prototype yields promising results, a browser extension will be developed as a follow-up to evaluate the scalability and effectiveness of the selected solutions in a realistic use context. This iterative approach enables a structured evaluation of technologies prior to full implementation.

The anticipated outcomes include improved accessibility and ease of use compared to traditional password managers, leading to increased user satisfaction and reduced frustration during authentication. This study aims to demonstrate the potential of facial recognition to address real-world accessibility barriers in digital identity management, thereby empowering users with disabilities to manage their credentials independently and securely.
