%%=============================================================================
%% Samenvatting
%%=============================================================================

% TODO: De "abstract" of samenvatting is een kernachtige (~ 1 blz. voor een
% thesis) synthese van het document.
%
% Een goede abstract biedt een kernachtig antwoord op volgende vragen:
%
% 1. Waarover gaat de bachelorproef?
% 2. Waarom heb je er over geschreven?
% 3. Hoe heb je het onderzoek uitgevoerd?
% 4. Wat waren de resultaten? Wat blijkt uit je onderzoek?
% 5. Wat betekenen je resultaten? Wat is de relevantie voor het werkveld?
%
% Daarom bestaat een abstract uit volgende componenten:
%
% - inleiding + kaderen thema
% - probleemstelling
% - (centrale) onderzoeksvraag
% - onderzoeksdoelstelling
% - methodologie
% - resultaten (beperk tot de belangrijkste, relevant voor de onderzoeksvraag)
% - conclusies, aanbevelingen, beperkingen
%
% LET OP! Een samenvatting is GEEN voorwoord!

%%---------- Nederlandse samenvatting -----------------------------------------
%
% TODO: Als je je bachelorproef in het Engels schrijft, moet je eerst een
% Nederlandse samenvatting invoegen. Haal daarvoor onderstaande code uit
% commentaar.
% Wie zijn bachelorproef in het Nederlands schrijft, kan dit negeren, de inhoud
% wordt niet in het document ingevoegd.

\IfLanguageName{english}{%
\selectlanguage{dutch}
\chapter*{Samenvatting}
De toenemende afhankelijkheid van online diensten heeft het belang van wachtwoordbeheer in digitale beveiliging versterkt. Traditionele authenticatiemethoden, zoals het invoeren en onthouden van complexe wachtwoorden, vormen vaak aanzienlijke uitdagingen voor personen met cognitieve of motorische beperkingen, waaronder moeilijkheden met geheugen of manuele input. Deze beperkingen onderstrepen de dringende nood aan toegankelijke authenticatieoplossingen die zowel de gebruiksvriendelijkheid als de tevredenheid van de gebruiker verbeteren.

De centrale onderzoeksvraag luidt: \emph{Hoe kan gezichtsherkenningstechnologie de toegankelijkheid van wachtwoordbeheer verbeteren voor personen met cognitieve en motorische beperkingen?}

Het doel van deze bachelorproef is het ontwerpen van een wachtwoordmanager op basis van gezichtsherkenning, specifiek gericht op personen met cognitieve of motorische beperkingen.

Het onderzoek omvat een vergelijkende analyse van verschillende gezichtsherkennings-API’s in diverse programmeertalen om de meest geschikte technologiestack te bepalen. Met de geselecteerde tools werd vervolgens een webgebaseerd prototype van een wachtwoordmanager ontwikkeld. Het prototype volgt een hybride architectuur waarbij de browser verantwoordelijk is voor gezichtsdetectie en client-side versleuteling van de beelden, terwijl de server instaat voor het extraheren van gezichtskenmerken, het uitvoeren van vergelijkingen en het veilig opslaan van de gegevens.

Indien het prototype veelbelovende resultaten oplevert, zal in een vervolgfase een browserextensie ontwikkeld worden om de schaalbaarheid en doeltreffendheid van de gekozen oplossingen in een realistische gebruiksomgeving te evalueren. Deze iteratieve aanpak maakt een gestructureerde evaluatie van de technologieën mogelijk vóór volledige implementatie.

De verwachte resultaten omvatten een verbeterde toegankelijkheid en gebruiksgemak ten opzichte van traditionele wachtwoordbeheerders, wat leidt tot een hogere gebruikerstevredenheid en minder frustratie tijdens het inloggen. Deze studie beoogt het potentieel van gezichtsherkenning aan te tonen om reële toegankelijkheidsproblemen in digitaal identiteitsbeheer aan te pakken, en zo personen met een beperking in staat te stellen hun inloggegevens zelfstandig en veilig te beheren.
\selectlanguage{english}
}{}

%%---------- Samenvatting -----------------------------------------------------
% De samenvatting in de hoofdtaal van het document

\chapter*{\IfLanguageName{dutch}{Samenvatting}{Abstract}}

The increasing reliance on online services has elevated the importance of password management in digital security. Traditional authentication methods, such as typing and remembering complex passwords, often pose significant challenges for individuals with cognitive or motor disabilities, including difficulties with memory recall or manual input. These limitations highlight the urgent need for accessible authentication solutions that improve both user satisfaction and usability.

The central research question is: \emph{How can face recognition technology improve accessibility in password management for individuals with cognitive and motor disabilities?} 

The objective of this thesis is to design a facial recognition-based password manager specifically aimed at individuals with cognitive or motor impairments.

The research includes a comparative analysis of several facial-recognition APIs across multiple programming languages to determine the best technology stack, and a web-based prototype password manager was then built with the selected tools. The prototype follows a hybrid architecture in which the browser executes face detection and client-side encryption of captured images, while the server performs descriptor extraction, matching, and secure storage.

If the prototype yields promising results, a browser extension will be developed as a follow-up to evaluate the scalability and effectiveness of the selected solutions in a realistic use context. This iterative approach enables a structured evaluation of technologies prior to full implementation.

The anticipated outcomes include improved accessibility and ease of use compared to traditional password managers, leading to increased user satisfaction and reduced frustration during authentication. This study aims to demonstrate the potential of facial recognition to address real-world accessibility barriers in digital identity management, thereby empowering users with disabilities to manage their credentials independently and securely.
