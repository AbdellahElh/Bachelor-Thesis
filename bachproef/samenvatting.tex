%%=============================================================================
%% Samenvatting
%%=============================================================================

% TODO: De "abstract" of samenvatting is een kernachtige (~ 1 blz. voor een
% thesis) synthese van het document.
%
% Een goede abstract biedt een kernachtig antwoord op volgende vragen:
%
% 1. Waarover gaat de bachelorproef?
% 2. Waarom heb je er over geschreven?
% 3. Hoe heb je het onderzoek uitgevoerd?
% 4. Wat waren de resultaten? Wat blijkt uit je onderzoek?
% 5. Wat betekenen je resultaten? Wat is de relevantie voor het werkveld?
%
% Daarom bestaat een abstract uit volgende componenten:
%
% - inleiding + kaderen thema
% - probleemstelling
% - (centrale) onderzoeksvraag
% - onderzoeksdoelstelling
% - methodologie
% - resultaten (beperk tot de belangrijkste, relevant voor de onderzoeksvraag)
% - conclusies, aanbevelingen, beperkingen
%
% LET OP! Een samenvatting is GEEN voorwoord!

%%---------- Nederlandse samenvatting -----------------------------------------
%
% TODO: Als je je bachelorproef in het Engels schrijft, moet je eerst een
% Nederlandse samenvatting invoegen. Haal daarvoor onderstaande code uit
% commentaar.
% Wie zijn bachelorproef in het Nederlands schrijft, kan dit negeren, de inhoud
% wordt niet in het document ingevoegd.

\IfLanguageName{english}{%
\selectlanguage{dutch}
\chapter*{Samenvatting}
De toenemende afhankelijkheid van online diensten heeft het belang van wachtwoordbeheer binnen digitale beveiliging versterkt. Traditionele authenticatiemethoden, zoals het intypen en onthouden van complexe wachtwoorden, vormen vaak aanzienlijke drempels voor mensen met cognitieve of motorische beperkingen, zoals geheugenproblemen of moeilijkheden bij het typen. Deze uitdagingen benadrukken de noodzaak van toegankelijke authenticatieoplossingen die de gebruikservaring en tevredenheid verhogen. Dit voorstel onderzoekt het potentieel van gezichtsherkenning als toegankelijk authenticatiemechanisme binnen wachtwoordbeheer, specifiek voor personen met cognitieve en motorische beperkingen.

De centrale onderzoeksvraag luidt: hoe kan gezichtsherkenning de toegankelijkheid van wachtwoordbeheer verbeteren voor mensen met cognitieve en motorische beperkingen? Het onderzoek omvat een analyse van diverse gezichtsherkennings-API’s in meerdere programmeertalen om de meest geschikte technologie en taal voor deze use-case te bepalen. Op basis van de bevindingen wordt een webgebaseerd prototype van een wachtwoordmanager ontwikkeld met de gekozen technologieën. Bij positieve resultaten volgt de ontwikkeling van een desktoptoepassing om de schaalbaarheid en doeltreffendheid van de geselecteerde oplossingen te evalueren. Deze iteratieve methodologie garandeert een goed onderbouwde en grondige evaluatie van de technologieën vóór volledige implementatie.

Verwachte uitkomsten zijn een verbeterde toegankelijkheid en gebruiksvriendelijkheid ten opzichte van traditionele wachtwoordmanagers, wat leidt tot meetbare stijgingen in gebruikerstevredenheid en minder frustratie tijdens het authenticatieproces. Deze studie beoogt aan te tonen dat gezichtsherkenning de toegankelijkheidsuitdagingen binnen wachtwoordbeheer kan aanpakken, zodat personen met cognitieve en motorische beperkingen hun digitale referenties zelfstandig en doeltreffend kunnen beheren.
\selectlanguage{english}
}{}

%%---------- Samenvatting -----------------------------------------------------
% De samenvatting in de hoofdtaal van het document

\chapter*{\IfLanguageName{dutch}{Samenvatting}{Abstract}}

The increasing reliance on online services has elevated the importance of password management in digital security. Traditional authentication methods, such as typing and remembering complex passwords, often pose significant challenges for individuals with cognitive or motor disabilities, such as difficulties with memory recall or typing. These challenges highlight the need for accessible authentication solutions that enhance user satisfaction and experience. This proposal examines the potential of integrating face recognition technology as an accessible authentication mechanism in password management, specifically for individuals with cognitive and motor disabilities.
The primary research question is: How can face recognition technology improve accessibility in password management for individuals with cognitive and motor disabilities? The research will involve analyzing various face recognition APIs across multiple programming languages to identify the most suitable technology and programming language for this use case. Based on the findings, a web-based prototype password manager will be developed using the chosen technologies. If successful, the next step will involve creating a desktop application to evaluate the scalability and effectiveness of the selected solutions. This iterative methodology ensures a well-informed and comprehensive evaluation of the technologies before full implementation.
Expected outcomes include improved accessibility and usability compared to traditional password managers, leading to measurable enhancements in user satisfaction and reduced frustration during authentication processes. This study aims to demonstrate the potential of face recognition technology to address accessibility challenges in password management, helping individuals with cognitive and motor disabilities to independently and effectively manage their digital credentials.

