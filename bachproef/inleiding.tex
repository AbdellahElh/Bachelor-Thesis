%%=============================================================================
%% Inleiding
%%=============================================================================

\chapter{\IfLanguageName{dutch}{Inleiding}{Introduction}}%
\label{ch:inleiding}

In today's digital age, authentication is a crucial element of cybersecurity. Traditional password-based systems require users to create, remember, and accurately enter complex passwords. For individuals with cognitive or motor disabilities, this process presents significant challenges that can lead to frustration, frequent lockouts, and increased security risks. This thesis investigates the integration of face recognition technology into a password manager to develop an accessible and user-friendly authentication method.

\subsection{Context and Background}
The increasing reliance on online services has amplified the need for secure and accessible authentication solutions. Conventional methods, such as passwords, often fail to accommodate the unique needs of users with disabilities. Cognitive impairments may hinder memory recall, while motor disabilities can complicate the physical act of typing. These challenges underscore the necessity for an alternative approach—one that minimizes cognitive and physical burdens without compromising security. In this context, the proposed research explores the use of face recognition technology as a viable solution to enhance digital authentication.

\subsection{Justification of the Topic}
This research focuses exclusively on evaluating face recognition as an alternative to traditional master passwords in a web-based password manager. Key points include:
\begin{itemize}
  \item \textbf{Target Group:} Individuals with cognitive and motor disabilities.
  \item \textbf{Tool Selection:} Emphasis on tools like \texttt{face-api.js}, which operate entirely in the browser to ensure privacy by eliminating server-side processing.
  \item \textbf{Alignment:} Integration with modern web technologies and adherence to accessibility standards.
\end{itemize}

\section{\IfLanguageName{dutch}{Probleemstelling}{Problem Statement}}%
\label{sec:probleemstelling}

Traditional password systems impose substantial cognitive and physical demands on users, particularly those with disabilities. The inherent complexity of creating, remembering, and entering passwords can significantly reduce user autonomy and lead to security vulnerabilities. This research seeks to determine how face recognition technology can replace traditional passwords in a way that overcomes these challenges—ultimately providing a more accessible and secure solution for the target group.

\section{\IfLanguageName{dutch}{Onderzoeksvraag}{Research Question}}%
\label{sec:onderzoeksvraag}

The central research question of this thesis is:
\begin{quote}
How can face recognition technology be effectively integrated into a password manager to improve accessibility for individuals with cognitive and motor disabilities?
\end{quote}

To explore this question further, the following sub-questions will be addressed:
\begin{enumerate}
  \item What specific challenges do individuals with cognitive and motor disabilities face with traditional password systems?
  \item How does face recognition technology compare to conventional methods in terms of usability and security?
  \item What are the technical requirements and potential obstacles for incorporating face recognition into existing password management systems?
\end{enumerate}

\section{\IfLanguageName{dutch}{Onderzoeksdoelstelling}{Research Objective}}%
\label{sec:onderzoeksdoelstelling}

The primary objective of this thesis is to design and develop a proof-of-concept web-based password manager that utilizes face recognition for user authentication. Success will be measured against the following criteria:
\begin{itemize}
  \item \textbf{Enhanced Accessibility:} Reducing the cognitive and physical effort required during authentication.
  \item \textbf{Improved Security:} Addressing vulnerabilities inherent in text-based passwords.
  \item \textbf{User-Friendly Interface:} Adhering to Web Content Accessibility Guidelines (WCAG) to ensure ease of use for the target audience.
\end{itemize}

\section{\IfLanguageName{dutch}{Opzet van deze bachelorproef}{Structure of this Bachelor Thesis}}%
\label{sec:opzet-bachelorproef}

% Het is gebruikelijk aan het einde van de inleiding een overzicht te
% geven van de opbouw van de rest van de tekst. Deze sectie bevat al een aanzet
% die je kan aanvullen/aanpassen in functie van je eigen tekst.

De rest van deze bachelorproef is als volgt opgebouwd:

In Hoofdstuk~\ref{ch:stand-van-zaken} wordt een overzicht gegeven van de stand van zaken binnen het onderzoeksdomein, op basis van een literatuurstudie.

In Hoofdstuk~\ref{ch:methodologie} wordt de methodologie toegelicht en worden de gebruikte onderzoekstechnieken besproken om een antwoord te kunnen formuleren op de onderzoeksvragen.

% TODO: Vul hier aan voor je eigen hoofstukken, één of twee zinnen per hoofdstuk

In Hoofdstuk~\ref{ch:conclusie}, tenslotte, wordt de conclusie gegeven en een antwoord geformuleerd op de onderzoeksvragen. Daarbij wordt ook een aanzet gegeven voor toekomstig onderzoek binnen dit domein.