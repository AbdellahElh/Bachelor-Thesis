%%=============================================================================
%% Conclusie
%%=============================================================================

\chapter{Discussion}%
\label{ch:conclusie}

% TODO: Trek een duidelijke conclusie, in de vorm van een antwoord op de
% onderzoeksvra(a)g(en). Wat was jouw bijdrage aan het onderzoeksdomein en
% hoe biedt dit meerwaarde aan het vakgebied/doelgroep? 
% Reflecteer kritisch over het resultaat. In Engelse teksten wordt deze sectie
% ``Discussion'' genoemd. Had je deze uitkomst verwacht? Zijn er zaken die nog
% niet duidelijk zijn?
% Heeft het onderzoek geleid tot nieuwe vragen die uitnodigen tot verder 
%onderzoek?

\section{Discussion}

\subsection{Answer to the research question.}
This thesis asked \emph{how face-recognition technology can be integrated into a password manager to improve accessibility for people with cognitive and motor disabilities}.  
The prototype shows that a \textbf{hybrid} pipeline can replace the master password without sacrificing security.  
(1) The browser captures a webcam image, (2) performs lightweight face detection to confirm user presence, (3) transmits an encrypted selfie to the server, and (4) the backend extracts a 128-dimensional descriptor, compares it with the stored template, and keeps both biometric data and credentials protected with end-to-end AES-256 encryption.  
By replacing a textual password with facial recognition, the system removes cognitive and physical barriers identified in Chapter~\ref{ch:stand-van-zaken}, while encrypted storage and server-side verification retain strong guarantees against unauthorized access.

\subsection{Contribution and added value.}
The work delivers an openly documented proof of concept that combines elements rarely found together in existing password managers:

\begin{itemize}
  \item \textbf{Privacy-aware hybrid biometrics:} Only a lightweight face-detection model runs in the browser; full descriptor extraction and matching occur on the server, where the environment is easier to secure and audit. Raw selfies are encrypted client-side before transit, so unencrypted biometrics never cross the wire.
  \item \textbf{End-to-end cryptography:} PBKDF2-derived AES-256 keys ensure that neither credentials nor biometric images are ever stored in plaintext, meeting the design goal of ``zero readable secrets outside RAM'' (see Section~\ref{sec:crypto}).
  \item \textbf{WCAG-aligned interface:} High-contrast colour palette, focus indicators, and enlarged click targets satisfy Level~AA success criteria, enabling day-to-day credential management for users with motor or cognitive impairments.
\end{itemize}

\subsection{Critical reflection.}
Initial planning assumed the entire face-recognition pipeline could run in the browser. This approach was technically \emph{feasible} but not optimal:  
\begin{itemize}
  \item \textbf{Security.} Moving descriptor extraction and comparison to the server adds a second line of defence; client-only verification could be bypassed on a compromised device.
  \item \textbf{Performance.} Large model downloads (10-15 MB) and 100-200 MB of browser RAM would impair older hardware, whereas a single server instance amortises that cost.
\end{itemize}


\subsection{Open questions and future work.}
\begin{enumerate}
  \item \textbf{Usability research:} How do users with cognitive or motor impairments perceive the hybrid approach over extended periods?
  \item \textbf{Environmental robustness:} What accuracy trade-offs arise under poor lighting or with face coverings, and how should thresholds adapt?
  \item \textbf{Spoof resistance in a hybrid pipeline:} Which liveness-detection techniques (e.g.\ randomised blink prompts, depth from stereo, device-sensor fusion) provide the best defence against presentation attacks when only anti-spoof checks run client-side?
\end{enumerate}

\subsection{Conclusion.}
The prototype demonstrates that face recognition can meaningfully lower authentication barriers while maintaining rigorous security. \\ 
A hybrid architecture that combines client-side detection, server-side verification, and fully encrypted storage offers a practical balance between usability, privacy, and threat resistance. \\ 
Future implementations should incorporate robust liveness detection, conduct wider empirical studies with the target demographic, and investigate an optional offline mode, all without weakening the security guarantees achieved here.\\
