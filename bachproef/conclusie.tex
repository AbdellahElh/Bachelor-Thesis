%%=============================================================================
%% Conclusie
%%=============================================================================

\chapter{Discussion}%
\label{ch:conclusie}

% TODO: Trek een duidelijke conclusie, in de vorm van een antwoord op de
% onderzoeksvra(a)g(en). Wat was jouw bijdrage aan het onderzoeksdomein en
% hoe biedt dit meerwaarde aan het vakgebied/doelgroep? 
% Reflecteer kritisch over het resultaat. In Engelse teksten wordt deze sectie
% ``Discussion'' genoemd. Had je deze uitkomst verwacht? Zijn er zaken die nog
% niet duidelijk zijn?
% Heeft het onderzoek geleid tot nieuwe vragen die uitnodigen tot verder 
%onderzoek?

\section{Discussion}

\paragraph{Answer to the research question.}
The thesis asked: \emph{How can face-recognition technology be effectively integrated into a password manager to improve accessibility for individuals with cognitive and motor disabilities?}
The prototype demonstrates that a fully client-side pipeline—camera capture, descriptor extraction via \texttt{face-api.js}, comparison against a locally stored template, and AES-256 encryption of all vaulted secrets can replace a conventional master password while keeping both biometric data and credentials on the user’s device.
This design removes the memory and typing requirements that make text-based authentication difficult for the target audience and therefore directly addresses the accessibility gap identified in Chapter 2.

\paragraph{Contribution and added value.}
The work contributes an openly documented proof-of-concept that combines three elements rarely found together in existing password managers:

\begin{itemize}
\item \textbf{Privacy-preserving biometrics:} all face-recognition inference runs entirely in the browser; no raw images or embeddings leave the device.
\item \textbf{End-to-end cryptography:} PBKDF2-derived AES keys and zero plaintext outside RAM fulfil the project goal of \`\`zero plaintext secrets outside the browser''.
\item \textbf{WCAG-aligned interface:} colour contrast, focus indicators, and enlarged click targets satisfy Level AA success criteria, making day-to-day credential management feasible for users with motor or cognitive impairments.
\end{itemize}

Together, these elements show that modern web technologies can deliver an accessible, hands-free login experience without forcing users to trust cloud services with either passwords or biometrics. For practitioners in accessible cybersecurity, the prototype serves as a reference architecture illustrating how biometric convenience, strong cryptography, and standards-based accessibility can coexist.

\paragraph{Critical reflection.}
The outcome aligns with initial expectations that client-side biometrics would lower cognitive and physical barriers; however, several limitations remain:

\begin{itemize}
\item \textbf{Empirical depth.} Functional tests and automated accessibility audits were completed, but no longitudinal user study with people who have disabilities has yet been conducted, so real-world effectiveness is still unverified.
\item \textbf{Environmental robustness.} Accuracy under low-light conditions, diverse camera qualities, and occlusions (e.g.\ face masks or tinted glasses) has not been systematically measured.
\item \textbf{Device reach.} The current implementation presumes a webcam-equipped desktop browser; adaptation to mobile or low-resource devices is untested.
\end{itemize}

\paragraph{Open questions and future directions.}
The project raises several research questions worth pursuing:

\begin{itemize}
\item Which fallback mechanisms (PIN, NFC token, or delegated recovery) best balance accessibility with security when the camera is unavailable?
\item Can optional, end-to-end-encrypted cloud sync be added without undermining the privacy guarantees that are central to the current design?
\item What liveness-detection techniques (e.g.\ challenge–response animations or sensor-fusion depth cues) offer the best spoof-resistance for an entirely client-side solution?
\end{itemize}

\paragraph{Conclusion.}
By showing a feasible path toward a WCAG-conform, privacy-first password manager guarded by a face instead of a master password, this work extends the accessible-authentication literature with a concrete and reproducible artefact. While further user studies and technical hardening are required, the results indicate that browser-based face recognition can meaningfully lower barriers to secure credential management for people with cognitive and motor disabilities, and thus merit deeper exploration in both academic and commercial settings.

