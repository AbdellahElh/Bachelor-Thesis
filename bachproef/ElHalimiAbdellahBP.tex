%===============================================================================
% LaTeX sjabloon voor de bachelorproef toegepaste informatica aan HOGENT
% Meer info op https://github.com/HoGentTIN/latex-hogent-report
%===============================================================================

\documentclass[english,dit,thesis]{hogentreport}

% TODO:
% - If necessary, replace the option `dit`' with your own department!
%   Valid entries are dbo, dbt, dgz, dit, dlo, dog, dsa, soa
% - If you write your thesis in English (remark: only possible after getting
%   explicit approval!), remove the option "dutch," or replace with "english".
\usepackage{xcolor}
\usepackage{listings}
\usepackage{caption} % For proper captions
\usepackage{float} % For H placement specifier
\usepackage{mdframed} % For controlling frames
\usepackage{lipsum} % For blind text, can be removed after adding actual content
\usepackage{placeins}
\usepackage{pifont}                % brings in \ding{}
\usepackage{array}            % for m{width} column
\newcommand{\cmark}{\textcolor{ForestGreen}{\ding{51}}}

% Custom colors for VS Code dark theme
\definecolor{vscodeBackground}{RGB}{30, 30, 30}
\definecolor{vscodeComment}{RGB}{106, 153, 85}
\definecolor{vscodeKeyword}{RGB}{86, 156, 214}
\definecolor{vscodeString}{RGB}{206, 145, 120}
\definecolor{vscodeType}{RGB}{78, 201, 176}
\definecolor{vscodeForeground}{RGB}{220, 220, 220}

% Configure listings for TypeScript code
\lstdefinelanguage{TypeScript}{
  keywords={break, case, catch, continue, debugger, default, delete, do, else, finally, for, function, if, in, instanceof, new, return, switch, this, throw, try, typeof, var, void, while, with, const, let, async, await, class, export, extends, import, super, interface, from},
  keywordstyle=\color{vscodeKeyword},
  ndkeywords={true, false, null, undefined, string, number, boolean, any, void},
  ndkeywordstyle=\color{vscodeType},
  identifierstyle=\color{vscodeForeground},
  sensitive=true,
  comment=[l]{//},
  morecomment=[s]{/*}{*/},
  commentstyle=\color{vscodeComment},
  stringstyle=\color{vscodeString},
  morestring=[b]',
  morestring=[b]"
}

% Define a completely borderless style for code listings
\definecolor{numberGray}{RGB}{100, 100, 100}
\definecolor{lineNumberBg}{RGB}{40, 40, 40}

% Configure main listing style
\definecolor{frameColor}{RGB}{40, 40, 40}

\lstset{
  backgroundcolor=\color{vscodeBackground},
  basicstyle=\small\ttfamily\color{vscodeForeground},
  frame=none,
  framerule=0pt,
  framesep=0pt,
  xleftmargin=15pt,
  xrightmargin=5pt,
  breaklines=true,
  breakatwhitespace=false,
  postbreak=\raisebox{0ex}[0ex][0ex]{\ensuremath{\hookrightarrow\space}},
  tabsize=2,
  numbers=left,
  numberstyle=\tiny\color{numberGray},
  numbersep=10pt,
  showspaces=false,
  showstringspaces=false,
  showtabs=false,
  captionpos=b,
  columns=flexible,
  keepspaces=true,
  aboveskip=1em,
  belowskip=1em
}

% Just use the existing listings setup without trying to redefine environments
\captionsetup[lstlisting]{labelsep=period, labelfont=bf, textfont=it}

% Create a completely borderless frame for listings
\surroundwithmdframed[hidealllines=true, backgroundcolor=vscodeBackground, innerleftmargin=15pt, innerrightmargin=5pt, skipabove=1em, skipbelow=1em]{lstlisting}

% Add a bit of spacing around listings
\lstset{
  aboveskip=1em,
  belowskip=1em
}
% No need to include enumitem package as it's already loaded in hogentreport.cls
% Enable inline lists feature directly after the document class
\makeatletter
\AtBeginDocument{%
  \@ifpackageloaded{enumitem}{%
    \SetEnumitemKey{inline}{itemjoin={,\ },after=.}%
    \newlist{enumerate*}{enumerate*}{3}%
    \setlist[enumerate*]{label=\arabic*.}%
  }{}%
}

%% Pictures to include in the text can be put in the graphics/ folder
\graphicspath{{../graphics/}}

%% For source code highlighting, requires pygments to be installed
%% Compile with the -shell-escape flag!
\usepackage[chapter]{minted}
%% If you compile with the make_thesis.{bat,sh} script, use the following
%% import instead:
% \usepackage[chapter,outputdir=../output]{minted}
\usemintedstyle{solarized-light}

%% Formatting for minted environments.
\setminted{%
    autogobble,
    frame=lines,
    breaklines,
    linenos,
    tabsize=4
}

%% Ensure the list of listings is in the table of contents
\renewcommand\listoflistingscaption{%
    \IfLanguageName{dutch}{Lijst van codefragmenten}{List of listings}
}
\renewcommand\listingscaption{%
    \IfLanguageName{dutch}{Codefragment}{Listing}
}
\renewcommand*\listoflistings{%
    \cleardoublepage\phantomsection\addcontentsline{toc}{chapter}{\listoflistingscaption}%
    \listof{listing}{\listoflistingscaption}%
}

% Other packages not already included can be imported here

%%---------- Document metadata -------------------------------------------------
% TODO: Replace this with your own information
\author{Abdellah El Halimi}
\supervisor{Dhr. A. De Witte}
\cosupervisor{Dhr. C. Dutoict}
% \title[Optionele ondertitel]%
%     {Titel van de bachelorproef}
\title{Accessible Password Management Using Face Recognition for Individuals with Cognitive and Motor Disabilities}
\academicyear{2024-2025}
\examperiod{1}
\degreesought{\IfLanguageName{dutch}{Professionele bachelor in de toegepaste informatica}{Bachelor of applied computer science}}
\partialthesis{false} %% To display 'in partial fulfilment'
%\institution{Internshipcompany BVBA.}

%% Add global exceptions to the hyphenation here
\hyphenation{back-slash}

%% The bibliography (style and settings are  found in hogentthesis.cls)
\addbibresource{bachproef.bib}            %% Bibliography file
\addbibresource{../voorstel/voorstel.bib} %% Bibliography research proposal
\defbibheading{bibempty}{}

%% Prevent empty pages for right-handed chapter starts in twoside mode
\renewcommand{\cleardoublepage}{\clearpage}

\renewcommand{\arraystretch}{1.2}

%% Content starts here.
\begin{document}

%---------- Front matter -------------------------------------------------------

\frontmatter

\hypersetup{pageanchor=false} %% Disable page numbering references

%% Render a Dutch outer title page if the main language is English
% \IfLanguageName{english}{%
%     %% If necessary, information can be changed here
%     \degreesought{Professionele Bachelor toegepaste informatica}%
%     \begin{otherlanguage}{dutch}%
%        \maketitle%
%     \end{otherlanguage}%
% }{}

%% Generates title page content
\maketitle
\hypersetup{pageanchor=true}

%%=============================================================================
%% Voorwoord
%%=============================================================================

\chapter*{\IfLanguageName{dutch}{Woord vooraf}{Preface}}%
\label{ch:voorwoord}

%% TODO:
%% Het voorwoord is het enige deel van de bachelorproef waar je vanuit je
%% eigen standpunt (``ik-vorm'') mag schrijven. Je kan hier bv. motiveren
%% waarom jij het onderwerp wil bespreken.
%% Vergeet ook niet te bedanken wie je geholpen/gesteund/... heeft

This thesis was written in the context of my graduation from the Applied Computer Science program at the University College Ghent. This thesis serves as an exploration into the implementation of a facial-recognition-based authentication system for password management.

The entire first semester of the 2024-2025 academic year, I worked on writing and creating the research proposal and implementing a basic authentication system that exclusively utilized facial recognition to evaluate the project's feasibility. The actual implementation of the final prototype, along with writing this thesis, began in the second semester and proved to be a substantial challenge.

I chose this topic based on personal experience with password managers, where I found it a bit annoying to repeatedly enter a master password in the browser while the mobile version of the same provider operated seamlessly with biometric authentication alone. This discrepancy inspired me to explore the integration of facial recognition for more user-friendly and secure access management.

It was reassuring when I discussed the concept with Professor Eric Verbeiren, who encouraged its potential usefulness. He expressed that such a system would benefit individuals, including himself, who struggle with remembering passwords. His feedback and encouragement further motivated me to pursue this topic and explore its practical implications.

I would like to express my gratitude to several individuals who have supported me throughout this journey. First and foremost, I extend my thanks to my 
% co-promoter, Christophe Dutoict, for his guidance, patience, and expertise. His insights and assistance were essential in overcoming technical challenges and refining the project’s implementation, especially during the registration process.

% I am also deeply grateful to my 
supervisor, Andreas De Witte, for his continuous support and constructive feedback throughout my research. His guidance has been helpful in shaping the direction of this thesis and improving its academic rigor.

Lastly, I would like to thank my friend, Jari Vancauwenberghe, for dedicating time to read my thesis and testing the prototype. His feedback and suggestions were greatly helpful in refining both the thesis and the project itself.

%%=============================================================================
%% Samenvatting
%%=============================================================================

% TODO: De "abstract" of samenvatting is een kernachtige (~ 1 blz. voor een
% thesis) synthese van het document.
%
% Een goede abstract biedt een kernachtig antwoord op volgende vragen:
%
% 1. Waarover gaat de bachelorproef?
% 2. Waarom heb je er over geschreven?
% 3. Hoe heb je het onderzoek uitgevoerd?
% 4. Wat waren de resultaten? Wat blijkt uit je onderzoek?
% 5. Wat betekenen je resultaten? Wat is de relevantie voor het werkveld?
%
% Daarom bestaat een abstract uit volgende componenten:
%
% - inleiding + kaderen thema
% - probleemstelling
% - (centrale) onderzoeksvraag
% - onderzoeksdoelstelling
% - methodologie
% - resultaten (beperk tot de belangrijkste, relevant voor de onderzoeksvraag)
% - conclusies, aanbevelingen, beperkingen
%
% LET OP! Een samenvatting is GEEN voorwoord!

%%---------- Nederlandse samenvatting -----------------------------------------
%
% TODO: Als je je bachelorproef in het Engels schrijft, moet je eerst een
% Nederlandse samenvatting invoegen. Haal daarvoor onderstaande code uit
% commentaar.
% Wie zijn bachelorproef in het Nederlands schrijft, kan dit negeren, de inhoud
% wordt niet in het document ingevoegd.

\IfLanguageName{english}{%
\selectlanguage{dutch}
\chapter*{Samenvatting}
De toenemende afhankelijkheid van online diensten heeft het belang van veilig wachtwoordbeheer verder vergroot. Traditionele authenticatiemethoden, zoals het onthouden en intypen van complexe wachtwoorden, vormen vooral voor mensen met cognitieve of motorische beperkingen een aanzienlijke barrière. Dit onderzoek verkent daarom het gebruik van gezichtsherkenning als toegankelijk authenticatiemechanisme binnen een wachtwoordmanager.

De centrale onderzoeksvraag luidt: \emph{hoe kan gezichtsherkenning de toegankelijkheid van wachtwoordbeheer verbeteren voor mensen met cognitieve en motorische beperkingen}? We analyseerden diverse gezichtsherkennings-API’s in meerdere programmeertalen en kozen uiteindelijk een hybride architectuur: de browser voert de face-detectie uit, versleutelt het beeld met AES-256 en stuurt het naar de server, waar de volledige biometrische verwerking (landmarks, descriptorextractie en matching) plaatsvindt. Zowel de 128-dimensionale descriptors als de versleutelde selfies worden veilig opgeslagen in een SQLite-database via Prisma ORM.

Op basis van deze keuze is een webgebaseerd prototype ontwikkeld. Indien dit positieve resultaten oplevert, volgt een browserextensie als vervolgstap om schaalbaarheid en effectiviteit in een realistische context te toetsen. Deze iteratieve aanpak maakt het mogelijk technologieën stapsgewijs en onderbouwd te evalueren voordat tot volledige implementatie wordt overgegaan.

Verwachte uitkomsten zijn een aantoonbare daling in cognitieve en fysieke belasting ten opzichte van traditionele wachtwoordmanagers en een hogere gebruikerstevredenheid, terwijl de hybride opzet sterke beveiliging biedt door serverside verificatie en volledige end-to-end-versleuteling.

\selectlanguage{english}
}{}

%%---------- Samenvatting -----------------------------------------------------
% De samenvatting in de hoofdtaal van het document

\chapter*{\IfLanguageName{dutch}{Samenvatting}{Abstract}}

The increasing reliance on online services has elevated the importance of password management in digital security. Traditional authentication methods, such as typing and remembering complex passwords, often pose significant challenges for individuals with cognitive or motor disabilities, including difficulties with memory recall or manual input. These limitations highlight the urgent need for accessible authentication solutions that improve both user satisfaction and usability.

This thesis explores the potential of integrating facial recognition technology as an accessible authentication mechanism within password management systems, specifically targeting individuals with cognitive and motor impairments.

The central research question is: \emph{How can face recognition technology improve accessibility in password management for individuals with cognitive and motor disabilities?} The research includes a comparative analysis of various facial recognition APIs across multiple programming languages to determine the most appropriate technology stack for this use case. Based on these findings, a web-based prototype of a password manager was developed using the selected technologies. The prototype follows a hybrid architecture in which the browser executes face detection and client-side encryption of captured images, while the server performs descriptor extraction, matching, and secure storage.

If the prototype yields promising results, a browser extension will be developed as a follow-up to evaluate the scalability and effectiveness of the selected solutions in a realistic use context. This iterative approach enables a structured evaluation of technologies prior to full implementation.

The anticipated outcomes include improved accessibility and ease of use compared to traditional password managers, leading to increased user satisfaction and reduced frustration during authentication. This study aims to demonstrate the potential of facial recognition to address real-world accessibility barriers in digital identity management, thereby empowering users with disabilities to manage their credentials independently and securely.


%---------- Inhoud, lijst figuren, ... -----------------------------------------

\tableofcontents

% In a list of figures, the complete caption will be included. To prevent this,
% ALWAYS add a short description in the caption!
%
%  \caption[short description]{elaborate description}
%
% If you do, only the short description will be used in the list of figures

\listoffigures

% If you included tables and/or source code listings, uncomment the appropriate
% lines.
\listoftables

\listoflistings

% Als je een lijst van afkortingen of termen wil toevoegen, dan hoort die
% hier thuis. Gebruik bijvoorbeeld de ``glossaries'' package.
% https://www.overleaf.com/learn/latex/Glossaries

%---------- Kern ---------------------------------------------------------------

\mainmatter{}

% De eerste hoofdstukken van een bachelorproef zijn meestal een inleiding op
% het onderwerp, literatuurstudie en verantwoording methodologie.
% Aarzel niet om een meer beschrijvende titel aan deze hoofdstukken te geven of
% om bijvoorbeeld de inleiding en/of stand van zaken over meerdere hoofdstukken
% te verspreiden!

%%=============================================================================
%% Inleiding
%%=============================================================================

\chapter{\IfLanguageName{dutch}{Inleiding}{Introduction}}%
\label{ch:inleiding}

In today's digital age, authentication is a crucial element of cybersecurity. Traditional password-based systems require users to create, remember, and accurately enter complex passwords. For individuals with cognitive or motor disabilities, this process presents significant challenges that can lead to frustration, frequent lockouts, and increased security risks. This thesis investigates the integration of face recognition technology into a password manager to develop an accessible and user-friendly authentication method.

\subsection{Context and Background}
The increasing reliance on online services has amplified the need for secure and accessible authentication solutions. Conventional methods, such as passwords, often fail to accommodate the unique needs of users with disabilities. Cognitive impairments may hinder memory recall, while motor disabilities can complicate the physical act of typing. These challenges underscore the necessity for an alternative approach—one that minimizes cognitive and physical burdens without compromising security. In this context, the proposed research explores the use of face recognition technology as a viable solution to enhance digital authentication.

\subsection{Justification of the Topic}
This research focuses exclusively on evaluating face recognition as an alternative to traditional master passwords in a web-based password manager. Key points include:
\begin{itemize}
  \item \textbf{Target Group:} Individuals with cognitive and motor disabilities.
  \item \textbf{Tool Selection:} Emphasis on tools like \texttt{face-api.js}, which operate entirely in the browser to ensure privacy by eliminating server-side processing.
  \item \textbf{Alignment:} Integration with modern web technologies and adherence to accessibility standards.
\end{itemize}

\section{\IfLanguageName{dutch}{Probleemstelling}{Problem Statement}}%
\label{sec:probleemstelling}

Traditional password systems impose substantial cognitive and physical demands on users, particularly those with disabilities. The inherent complexity of creating, remembering, and entering passwords can significantly reduce user autonomy and lead to security vulnerabilities. This research seeks to determine how face recognition technology can replace traditional passwords in a way that overcomes these challenges—ultimately providing a more accessible and secure solution for the target group.

\section{\IfLanguageName{dutch}{Onderzoeksvraag}{Research Question}}%
\label{sec:onderzoeksvraag}

The central research question of this thesis is:
\begin{quote}
How can face recognition technology be effectively integrated into a password manager to improve accessibility for individuals with cognitive and motor disabilities?
\end{quote}

To explore this question further, the following sub-questions will be addressed:
\begin{enumerate}
  \item What specific challenges do individuals with cognitive and motor disabilities face with traditional password systems?
  \item How does face recognition technology compare to conventional methods in terms of usability and security?
  \item What are the technical requirements and potential obstacles for incorporating face recognition into existing password management systems?
\end{enumerate}

\section{\IfLanguageName{dutch}{Onderzoeksdoelstelling}{Research Objective}}%
\label{sec:onderzoeksdoelstelling}

The primary objective of this thesis is to design and develop a proof-of-concept web-based password manager that utilizes face recognition for user authentication. Success will be measured against the following criteria:
\begin{itemize}
  \item \textbf{Enhanced Accessibility:} Reducing the cognitive and physical effort required during authentication.
  \item \textbf{Improved Security:} Addressing vulnerabilities inherent in text-based passwords.
  \item \textbf{User-Friendly Interface:} Adhering to Web Content Accessibility Guidelines (WCAG) to ensure ease of use for the target audience.
\end{itemize}

\section{\IfLanguageName{dutch}{Opzet van deze bachelorproef}{Structure of this Bachelor Thesis}}
\label{sec:opzet-bachelorproef}

The remainder of this bachelor thesis is organised as follows:

\begin{description}
  \item[Chapter~\ref{ch:stand-van-zaken} \textemdash\ State of the Art] provides a targeted literature review.  
        It surveys traditional and biometric authentication methods,  
        analyses the specific accessibility challenges these methods pose,  
        reviews face-recognition algorithms and libraries,  
        compares leading JavaScript, Python and Java implementations, and  
        summarises security, cryptographic and WCAG 2.1 accessibility requirements that shape the remainder of the work.

  \item[Chapter~\ref{ch:methodologie} \textemdash\ Methodology] explains the research design and technical approach.  
        It justifies the selection of \texttt{face-api.js}, details the agile, TypeScript-based prototype development process, motivates the choice of SQLite for local credential storage, and describes the success metrics (accessibility gain, security strength and usability) that will later be evaluated.

  \item[Chapter~\ref{ch:implementatie} \textemdash\ Prototype Implementation] documents the architecture of the web-based password manager.  
        It covers the facial-authentication workflow, from backend to frontend; the cryptographic handling of credentials with \texttt{bcrypt}, and the WCAG-compliant user interface.

  \item[Chapter~\ref{ch:conclusie} \textemdash\ Conclusion] reports the empirical results and synthesises the findings. 
        It presents functional tests and an accessibility study. The outcomes are discussed against the success metrics limitations are reflected upon, and avenues for future work are proposed.
\end{description}

% filepath: d:\abdey\Documents\3. Toegepaste info\BachProef\latex-hogent-bachproef-main\latex-hogent-bachproef-main\bachproef\standvanzaken.tex
\chapter{\IfLanguageName{dutch}{Stand van zaken}{State of the art}}%
\label{ch:stand-van-zaken}

% Tip: Begin elk hoofdstuk met een paragraaf inleiding die beschrijft hoe
% dit hoofdstuk past binnen het geheel van de bachelorproef. Geef in het
% bijzonder aan wat de link is met het vorige en volgende hoofdstuk.

% Pas na deze inleidende paragraaf komt de eerste sectiehoofding.


\section{Authentication in Digital Security}
Authentication is the process of verifying the identity of users attempting to access digital systems or online services. 
Commonly used methods include knowledge-based authentication (passwords), possession-based methods (tokens), and 
biometric methods like fingerprints and facial recognition \autocite{Pant2022}. Passwords remain dominant due to 
their simplicity and widespread acceptance, but face security issues including password reuse, phishing, and 
brute-force attacks \autocite{Ophoff2021}. For users with cognitive or motor disabilities, these issues are 
further complicated by difficulties in recalling or accurately inputting passwords \autocite{Rochford2014}.\\
These observations answer sub-question~\ref{sq:challenges} by clarifying which
cognitive and motor barriers the new solution must overcome.


\section{Accessibility Challenges in Authentication}
Individuals with cognitive disabilities, such as memory disorders or conditions like dyslexia, often struggle with remembering complex passwords, resulting in frequent authentication failures and frustration \autocite{Farid2019, Ophoff2021}. Those with motor disabilities, including conditions like Parkinson's disease or cerebral palsy, face physical challenges in typing passwords accurately \autocite{Renaud2020}. The Web Content Accessibility Guidelines (WCAG) highlight the importance of designing authentication systems that minimize these cognitive and physical burdens \autocite{Brewer2023}.

\clearpage

\section{Current Limitations in Password Managers}
Password managers simplify password management by securely storing and auto-filling credentials but commonly rely on a master password, perpetuating cognitive and motor accessibility issues. This approach is problematic for users who struggle with memory recall or precise typing \autocite{IALabs2024}. While MFA offers increased security, it often introduces additional complexity that further burdens users with disabilities. Current systems have limited inclusivity and accessibility, reinforcing the need for more intuitive solutions.

\section{Facial Recognition Technology}
Facial recognition works by identifying and verifying individuals from digital images or videos using various algorithmic approaches, including traditional image processing methods and modern deep learning techniques. Notable algorithms include Haar cascades, Eigenfaces, Local Binary Patterns Histograms (LBPH), and Convolutional Neural Networks (CNNs) \autocite{ElSayed2015}. The evolution of deep learning, particularly CNN-based approaches, has significantly enhanced accuracy and reliability, making facial recognition robust even under challenging conditions like variations in lighting, angle, or facial expressions \autocite{ZhangDlib2020}.\\


\section{Face Recognition as a Biometric Solution}
Biometric authentication, particularly facial recognition, is gaining popularity as it significantly reduces the cognitive and physical effort required by traditional password-based methods \autocite{Furnell2022}. 
Unlike passwords, biometric data are unique physical attributes of an individual, providing an inherent security advantage by eliminating risks associated with knowledge-based authentication methods, 
such as forgetting or sharing passwords \autocite{Pant2022}.

Facial recognition stands out as particularly promising because it is intuitive, does not require manual input, and can be seamlessly integrated into daily digital interactions \autocite{Bhatt2011}. However, biometric systems are not without limitations. Spoofing attacks, privacy concerns, and the requirement for consistent lighting and camera quality present technical and ethical considerations that must be carefully managed \autocite{Kuznetsov2024, Bahia2024}.

\clearpage

\subsection{Comparison of Facial Recognition Libraries}

\subsubsection{OpenCV (Java)}
OpenCV is a widely used open-source library offering classical computer vision techniques such as Haar cascades and LBPH. While effective in controlled environments, it typically requires server-side or desktop-based implementations, limiting its applicability for client-side web applications \autocite{Dominguez2017}.

\subsubsection{face\_recognition in Python}  
The face\_recognition library, built on dlib, is renowned for its accuracy and pre-trained deep learning models. \textcite{ZhangDlib2020} highlight that it excels in applications requiring precision, utilizing techniques like CNN-based face encodings for high-quality results. However, its reliance on Python and backend processing makes it less suitable for client-side, browser-based implementations like those required in this project.

\subsubsection{face-api.js (JavaScript)}
face-api.js, built on TensorFlow.js, runs entirely in the browser, providing a privacy-centric, client-side solution suitable for real-time applications \autocite{Vageele2024}. Its key benefits include privacy (no server-side data transfer), compatibility with modern web frameworks, and modularity for lightweight and efficient real-time processing. These features align closely with the project's emphasis on usability, accessibility, and security, making face-api.js the optimal choice for this research.

\section{Security Considerations for Biometric Authentication}
While facial recognition enhances accessibility, it also introduces specific security concerns. Common vulnerabilities include spoofing attacks using photos or video recordings and data privacy issues related to biometric data storage \autocite{Bowyer2006, Bahia2024}. Beyond these well-known threats, systems that store facial biometric data face additional sophisticated attack vectors:

\begin{itemize}
\item \textbf{Template Extraction Attacks:} These attacks aim to reconstruct biometric templates from stored embeddings, potentially compromising the entire authentication system. Even when templates are encrypted, side-channel attacks can leak information about the underlying biometric data \autocite{Mai2019, Dong2021}. As demonstrated by \textcite{Mai2019}, deep face templates previously thought to be secure can be reverse-engineered to recover recognizable face images, raising significant privacy concerns. Building on this work, \textcite{Dong2021} showed that high-definition face images can be generated from supposedly secure templates using advanced generative models.

\clearpage

\item \textbf{Model Inversion Attacks:} Adversaries can exploit machine learning models to regenerate facial images from stored embeddings or model parameters, effectively reversing the feature extraction process \autocite{Fredrikson2015, Zhang2020}. \textcite{Fredrikson2015} demonstrated that these attacks can leverage confidence information from ML models to reconstruct training data, while \textcite{Zhang2020} advanced this concept with generative model-inversion attacks against deep neural networks. This poses significant privacy risks as it may allow attackers to recreate recognizable facial images from supposedly secure numerical representations.

\item \textbf{Spoofing Attacks:} Presentation attacks using photos or video recordings of legitimate users to deceive recognition systems \autocite{Kuznetsov2024}.
\end{itemize}

Modern mitigation strategies include:
\begin{itemize}
\item \textbf{Liveness Detection:} Techniques to ensure the presence of a real, live user rather than a static image or video \autocite{Kuznetsov2024}.

\item \textbf{Client-side Encryption:} Captured images are encrypted in the browser before transmission, so no unencrypted biometric data leaves the device even though full descriptor extraction is performed server-side.

\item \textbf{Multi-factor Authentication (MFA):} Combining biometric data with other authentication methods to provide layers of security and protect against vulnerabilities inherent in single-method authentication systems \autocite{Furnell2022}.

\item \textbf{Cancelable Biometrics:} Applying irreversible transformations to biometric templates before storage, allowing for template revocation if compromised. \textcite{Rathgeb2011} provide a comprehensive survey of these techniques, highlighting their importance in protecting biometric data while maintaining authentication accuracy.
\end{itemize}

\clearpage


\section{Database Options for Client\textendash Side Password Storage}

A password manager must choose a local datastore that balances footprint,
offline capability, security, and future synchronisation needs.  The five
candidates below are summarised with literature references and official
documentation links.

\subsection*{SQLite}
\textcite{Gaffney2022} show that SQLite embeds the entire ACID-compliant
engine in a single file of only a few-hundred-kB, requiring no server
process.  Its dynamic typing allows flexible schemas \autocite{Corovcak2025}.
Because it lacks native user authentication, security depends on OS file
permissions or extensions such as SQLCipher \autocite{Corovcak2025}.
Official documentation confirms the zero-config model and SQL feature set
\autocite{sqlLiteDoc2025}. These traits make SQLite ideal for an offline,
single-device password vault, provided the file is encrypted at rest.

\subsection*{PostgreSQL}
PostgreSQL's client-server design provides robust concurrency and rich SQL
features after 35-years of development \autocite{Gkamas2022}. It provides 
built-in role-based access control and supports TLS to secure data in transit.
However, the community edition lacks native encryption at rest, so administrators 
must rely on the \textit{pgcrypto} extension or file-system encryption 
\autocite{Crunchy2024, PostgreSQL2025}. A local instance typically consumes hundreds of-MB of RAM,
which is heavy for a single-user password vault. Hence PostgreSQL is secure and scalable,
but overkill for a single-user application.

\subsection*{MongoDB}
MongoDB stores JSON-like documents in flexible collections and scales
horizontally \autocite{Miryala2024}.  A \texttt{mongod} process needs 1-2-GB
RAM even for modest use \autocite{Dahunsi2021}.  Community builds provide
authentication and TLS, yet encryption at rest is Enterprise-only
\autocite{PrismaMongoEnc, MongoDB2025}, so disk encryption or field-level
crypto is required.  Misconfiguration has repeatedly exposed databases,
underscoring the need for hardened defaults.  The
server footprint makes MongoDB unsuitable for a secure, single-user password vault.

\clearpage

\subsection*{Couchbase Lite}
Couchbase Lite embeds a document store inside the app and syncs through
Sync Gateway when online \autocite{Pal2016}.  Its metadata inflates on-disk
size versus SQLite, yet runtime demands remain mobile-friendly
\autocite{Gkamas2022}.  The Enterprise build supports 256-bit AES encryption
of the local DB \autocite{CouchbaseEncryption, CouchbaseDoc2025}.  Because it
executes in the app's sandbox, further authentication is handled by the host
application.  These qualities make Couchbase Lite attractive for an
offline-first vault that may later sync across devices.

\subsection*{Firebase Cloud Firestore}
Firestore is a serverless NoSQL service that caches data locally and
synchronises transparently once connectivity returns \autocite{FirebaseDoc2025}.
Security combines Firebase Authentication with declarative Firestore Rules,
while Google encrypts data in transit and at rest \autocite{FirebaseSecurity2025}.
This offloads database maintenance but requires Internet access for initial
login and long-term storage.  Firestore therefore suits a multi-device,
cloud-centric password manager but {\fontseries{sb}\selectfont cedes full data custody to Google}.


\section{Cryptographic Security in Password Managers}
\label{sec:crypto}
Secure storage of credentials is fundamental in password management.
In this project, cryptographic operations execute client-side with the
\texttt{crypto-js} library \autocite{CryptoJS2024}, combining AES-256 for
confidentiality and PBKDF2 for key derivation.  AES supplies a
NIST-approved block cipher \autocite{NISTFIPS197}, while PBKDF2's 10\,000
iterations and per-installation salt greatly raise the cost of brute-force
attacks \autocite{RFC8018}.  Keeping both encryption and decryption in the
browser ensures plaintext credentials or biometric images never leave the
device, aligning with OWASP key-management guidance
\textcite{OWASPKeyMgmt2025}.

\subsubsection{Crypto-js}  
The \texttt{crypto-js} library offers JavaScript implementations of AES,
PBKDF2, and other standard algorithms through a concise API optimised for the
browser.  Its widespread adoption and open-source governance mean the code
base is continuously scrutinised and updated for vulnerabilities
\autocite{CryptoJS2024}.

\subsubsection{AES}  
AES-256 encrypts both passwords and face images, providing a large key space
and proven resistance to cryptanalysis.  Defined in FIPS 197, AES remains the standard for protecting electronic data across government and
industry \autocite{NISTFIPS197}. 

\clearpage

\subsubsection{Encryption Key Handling}  
Each encryption key is derived in the browser at login and kept only in
memory; it is never persisted or sent to the backend.  This client-centric
approach follows OWASP key-management recommendations, ensuring a server
breach alone cannot expose decryption keys
\autocite{OWASPKeyMgmt2025}. 

\subsubsection{PBKDF2}  
PBKDF2, specified in RFC 8018, transforms the user's secret into a strong
256-bit key using 10\,000 iterations and a unique 16-byte salt.  Iterative
key stretching slows dictionary attacks, while the salt thwarts rainbow
tables \autocite{RFC8018}.\\

This technologies answers sub-question~\ref{sq:implementation}, by detailing
which recognition approaches are sufficiently accurate and performant for
a web-based password manager.

\section{Performance Benchmarks in Facial Biometrics}
\label{sec:biometric-benchmarks}

It is important to evaluate the security accuracy of the facial authentication. While a full-scale laboratory test of false acceptance and rejection rates is beyond the scope of this study, published benchmarks from similar biometric systems provide reference performance targets for the prototype in terms of False Accept Rate (FAR) and False Reject Rate (FRR).

\textbf{Windows Hello (IR Facial Login):} Microsoft's Windows Hello face authentication system, which uses infrared imaging, sets stringent hardware requirements with a FAR of less than 0.001\% (i.e., fewer than 1 in 100{,}000 impostor attempts succeed) \autocite{MicrosoftHelloDocs}. In practice, Microsoft reports a FRR under 5\% without liveness detection, and under 10\% when anti-spoofing measures are enabled \autocite{MicrosoftHelloDocs}. The web-based prototype (using a standard camera and \texttt{face-api.js}) cannot match these hardware-accelerated standards, but follows the same design philosophy: prioritizing a low FAR even at the expense of occasional false rejects.

\textbf{Apple Face ID (3D Face Recognition):} Apple's Face ID system also uses 3D infrared facial imaging and targets an extremely low FAR, citing less than 1 in 1{,}000{,}000 chances of unauthorized access \autocite{BentoFaceID}. Although Apple has not disclosed detailed FRR statistics, empirical data suggests that its performance remains highly accurate for legitimate users in normal conditions, despite occasional false rejects caused by sunglasses or facial obstructions. As such, high-end consumer systems aim for FAR values between $10^{-5}$ and $10^{-6}$, while keeping FRR below 10\%.

\clearpage
\textbf{Academic Benchmarks (FRVT and Others):} Independent evaluations, such as the U.S. National Institute of Standards and Technology (NIST) Face Recognition Vendor Test (FRVT), provide authoritative insight. A top-performing algorithm in FRVT 1:1 verification achieved a false non-match rate of 0.36\% at a FAR of 1 in 1{,}000{,}000 \autocite{ParavisionFRVT}. These results represent optimal conditions; performance degrades in less-controlled environments such as those encountered by webcam-based authentication. Nonetheless, they indicate the capabilities of modern face recognition technologies.

\textbf{Typical Web/Mobile Authentication:} For systems using standard RGB cameras, face recognition APIs such as Microsoft Azure Face and Amazon Rekognition typically report over 99\% true acceptance rates under good conditions \autocite{IJCAFace}. Benchmarks on datasets like LFW (Labeled Faces in the Wild) indicate that many models achieve 99.5\%+ identification accuracy in uncontrolled photographic conditions. The UK National Cyber Security Centre estimates a FAR better than 0.1\% and FRR between 5--10\% for mobile face unlock systems \autocite{BentoFaceID,MicrosoftHelloDocs}.

Based on these references, the prototype aims for a FAR in the range of $\leq$0.1\% to prevent unauthorized access, while tolerating a FRR of 5--10\% as a reasonable usability trade-off. The matching threshold is set conservatively to favor security, accepting that users may occasionally need to retry under suboptimal conditions.

\section{Usability and Accessibility Standards}
Accessibility in digital solutions adheres to guidelines such as WCAG, which outline best practices for minimizing cognitive load, ensuring interface clarity, and reducing physical input requirements. Adopting these standards ensures the password manager prototype remains usable for individuals with various disabilities \autocite{Brewer2023}. Inclusive design principles further emphasize the need to involve users with disabilities in the development process to validate and refine usability \autocite{Lazar2015}.

%%=============================================================================
%% Methodologie
%%=============================================================================

\chapter{\IfLanguageName{dutch}{Methodologie}{Methodology}}%
\label{ch:methodologie}

The development process for this project is structured into distinct stages, each pursuing specific objectives, generating clear deliverables, and applying targeted research methods. The stages appear below.

\section{WCAG 2.1 Compliance Implementation}
\label{sec:wcag-compliance}

A comprehensive approach to accessibility requires systematic mapping of features against established standards. The tables below provide a traceability matrix that maps the password manager's features to the Web Content Accessibility Guidelines (WCAG) 2.1 success criteria. This matrix serves as both a design reference and a verification tool to ensure all accessibility requirements are addressed in the implementation phase.

\begin{table}[htbp]
  \centering
  \small
  % increase row height for breathing room
  \renewcommand{\arraystretch}{1.4}
  \begin{tabular}{|
      p{2.5cm}|
      p{1.5cm}|
      p{4cm}|
      p{4cm}|
      >{\centering\arraybackslash}m{1cm}|}
    \hline
    \textbf{WCAG 2.1} & \textbf{Conform-} & \textbf{Features} & \textbf{Evidence / Implementation} & \textbf{Status} \\ 
    \textbf{Success} & \textbf{ance} & & & \\ 
    \textbf{Criterion} & \textbf{Level} & & & \\ \hline
    
    1.1.1 Non-text Content 
      & A 
      & Face-registration and authentication UI, credential icons, action buttons 
      & alt text for static images; ARIA labels (e.g.\ \texttt{aria-label="Capture selfie"}) 
      & \cmark \\ \hline
    
    1.3.1 Info \& Relationships 
      & A 
      & Modal forms for Add/Edit Credential, list view of saved credentials 
      & Semantic HTML (\texttt{<form>}, \texttt{<label>}, \texttt{<ul>/<li>}); programmatic labels 
      & \cmark \\ \hline
    
    1.3.2 Meaningful Sequence 
      & A 
      & React component hierarchy for registration \& login flows 
      & DOM order matches visual order; logical tab order verified with keyboard 
      & \cmark \\ \hline
    
    1.3.4 Orientation 
      & AA 
      & Responsive layout (CSS Flex/Grid) 
      & UI adapts to portrait and landscape on mobile; no fixed-orientation lock 
      & \cmark \\ \hline
    
    1.4.3 Contrast (Minimum) 
      & AA 
      & Global Tailwind theme 
      & Colour palette tested with WCAG contrast checker: all combos $\geq$4.5:1 
      & \cmark \\ \hline
    
    1.4.5 Images of Text 
      & AA 
      & Credential list, buttons 
      & No rasterised text; all labels are true text 
      & \cmark \\ \hline
    
    1.4.10 Reflow 
      & AA 
      & Responsive UI on small screens 
      & Tested down to 320 px width without horizontal scroll 
      & \cmark \\ \hline
    
    1.4.11 Non-text Contrast 
      & AA 
      & Icon buttons, focus rings 
      & SVG icons meet 3:1; focus outline uses high-contrast colour 
      & \cmark \\ \hline
    
    2.1.1 Keyboard 
      & A 
      & All interactive controls 
      & Tab order flows through modals; face-capture via Enter key 
      & \cmark \\ \hline
    
    2.1.2 No Keyboard Trap 
      & A 
      & Modal dialogs 
      & Esc, Tab, Shift+Tab correctly move focus or close modal 
      & \cmark \\ \hline
  \end{tabular}
  \caption[WCAG 2.1 Traceability Matrix]{WCAG 2.1 Traceability Matrix for the Face-Based Password Manager.}
  \label{tab:wcag-matrix}
\end{table}

\begin{table}[htbp]
  \centering
  \small
  \renewcommand{\arraystretch}{1.4}
  \begin{tabular}{|
      p{2.5cm}|
      p{1.5cm}|
      p{4cm}|
      p{4cm}|
      >{\centering\arraybackslash}m{1cm}|}
    \hline
    \textbf{WCAG 2.1} & \textbf{Con-} & \textbf{Features} & \textbf{Evidence / Implementation} & \textbf{Status} \\ 
    \textbf{Success} & \textbf{formance} & & & \\ 
    \textbf{Criterion} & \textbf{Level} & & & \\ \hline
    
    2.2.1 Timing Adjustable 
      & A 
      & Face-capture countdown 
      & 5 s default; user can extend to 15 s 
      & \cmark \\ \hline
    
    2.3.1 Three Flashes or Below 
      & A 
      & Toast animations 
      & All animations <3 flashes/sec; none bright 
      & \cmark \\ \hline
    
    2.4.3 Focus Order 
      & A 
      & Credential list, modals 
      & Logical sequential focus verified with screen reader 
      & \cmark \\ \hline
    
    2.4.4 Link Purpose 
      & A 
      & External links in footer/help 
      & Link text describes destination (e.g.\ "WCAG Quick-Ref") 
      & \cmark \\ \hline
    
    2.4.7 Focus Visible 
      & AA 
      & Custom focus outline 
      & 2 px outline + 4 px offset for visibility 
      & \cmark \\ \hline
    
    3.2.1 On Focus 
      & A 
      & Input fields, buttons 
      & No unexpected context changes on focus 
      & \cmark \\ \hline
    
    3.2.2 On Input 
      & A 
      & Credential forms 
      & Only explicit Save triggers state change; live validation announced 
      & \cmark \\ \hline
    
    3.3.1 Error Identification 
      & A 
      & Form validation 
      & Inline errors exposed via \texttt{aria-describedby} 
      & \cmark \\ \hline
    
    3.3.2 Labels or Instructions 
      & A 
      & All form inputs, face capture 
      & Clear labels (e.g.\ "Email Address"), helper text 
      & \cmark \\ \hline
    
    4.1.1 Parsing 
      & A 
      & React front-end templates 
      & JSX compiles to valid HTML; no duplicate IDs 
      & \cmark \\ \hline
    
    4.1.2 Name, Role, Value 
      & A 
      & Custom components (Modal, Button, Toast) 
      & ARIA roles/states (e.g.\ \texttt{role="dialog"}, \texttt{aria-modal="true"}) 
      & \cmark \\ \hline
  \end{tabular}
  \caption[WCAG 2.1 Traceability Matrix (continued)]{WCAG 2.1 Traceability Matrix (continued).}
  \label{tab:wcag-matrix-cont}
\end{table}

\FloatBarrier

\section{Tool Selection}
This stage identifies the libraries and frameworks that best balance accessibility, security and ease of integration for a browser-based, face-recognition password manager, informed by the WCAG requirements established above.

\subsection*{Core Facial-Recognition Library}
\begin{itemize}
  \item \textbf{face-api.js}. In the browser we load only \texttt{ssdMobilenetv1} to check
  that a face is present. The encrypted frame is then uploaded,
  decrypted on the server, and the full models
  (\texttt{tinyFaceDetector}, \texttt{faceLandmark68Net},
  \texttt{faceRecognitionNet}) run there to extract the 128-D
  descriptor.  No plaintext biometric image is stored.

  \item Alternatives such as \textbf{OpenCV (Java)} and \textbf{face\_recognition (Python)} were benchmarked, yet \texttt{face-api.js} offered the best balance of privacy, performance and client-side deployment convenience for this use case.
\end{itemize}

\subsection*{Front-End Stack}
\begin{itemize}
  \item \textbf{React} \& \textbf{TypeScript}. Provide a component-based architecture with static typing that improves maintainability, refactoring safety and long-term scalability.
  \item \textbf{Vite}. A fast build tool and dev server that offers near-instant hot-module reloading, accelerating iterative UI development.
  \item \textbf{Tailwind CSS}. Utility-first styling ensures a consistent design system and simplifies compliance with WCAG colour-contrast and spacing requirements.
  \item \textbf{Axios} \& \textbf{SWR}. Axios handles HTTP requests; SWR ("stale-while-revalidate") supplies efficient client-side caching and auto-revalidation, improving perceived performance.
  \item \textbf{React-Hot-Toast} \& \textbf{React-Icons}. Lightweight libraries for accessible status notifications and vector icons, enhancing feedback without heavy dependencies.
\end{itemize}

\subsection*{Back-End Stack}
\begin{itemize}
  \item \textbf{Node.js \& Express}. A minimal, well-known HTTP framework that pairs naturally with a TypeScript codebase and supports rapid prototyping of REST endpoints.
  \item \textbf{Prisma ORM}. Provides type-safe data access, simplifying schema evolution and reducing boilerplate SQL queries.
  \item \textbf{SQLite}. An embedded ACID-compliant database ideal for a single-user vault; encrypts easily with SQLCipher if required, and avoids the overhead of a client–server engine. (See Section~\ref{sec:database-selection} for a detailed comparison with alternative databases.)
  \item \textbf{jsonwebtoken}. Issues and verifies stateless access tokens for authenticated API calls.
  \item \textbf{Multer}. Handles multipart uploads (encrypted face images) without blocking the event loop.
\end{itemize}

\subsection*{Client-Side Cryptography}
\begin{itemize}
  \item \textbf{Crypto-JS}. Implements industry-standard primitives directly in the browser. We use \textbf{AES-256} for confidentiality and \textbf{PBKDF2} for key derivation, as justified in the literature review (see Section~\ref{sec:crypto}) where their security properties are discussed in detail.
\end{itemize}

\subsection*{Development Tooling}
\begin{itemize}
  \item \textbf{ESLint} \& \textbf{typescript-eslint}. Enforce a consistent code style and surface common pitfalls early.
  \item \textbf{ts-node} \& \textbf{ts-node-dev}. Provide rapid TypeScript execution with automatic reloads during back-end development.
  \item \textbf{Concurrently}. Runs front-end and back-end watch tasks in parallel, streamlining the full-stack developer experience.
\end{itemize}

\subsection*{Why This Stack?}
Together, these tools deliver:

\begin{itemize}
  \item \textbf{End-to-End Privacy}. All biometric data are processed and encrypted client-side; only descriptors and ciphertext reach the server.
  \item \textbf{Accessibility by Design}. React, Tailwind and face-api.js enable responsive layouts, clear focus states and real-time feedback, directly supporting WCAG criteria.
  \item \textbf{Security Hardening}. Proven cryptographic primitives (AES-256, PBKDF2, bcrypt) and JWT-based sessions protect stored credentials and transport channels.
  \item \textbf{Developer Velocity}. A TypeScript mono-repo with hot reloading, linting and typed ORM reduces context switching and speeds iteration.
\end{itemize}

\section{Database Selection}
\label{sec:database-selection}
Choosing an efficient, secure database to manage user credentials forms a crucial component of the project. After evaluating multiple options against performance, security, scalability and integration criteria, the project currently employs:

\begin{itemize}
  \item \textbf{SQLite:} Adopted for the initial prototype because of its simplicity, lightweight footprint and suitability for local client-side storage.
  \item \textbf{PostgreSQL:} Evaluated for future scenarios that demand greater scalability, flexible data models and larger datasets.
\end{itemize}

Both MongoDB and Couchbase Lite were evaluated but not carried forward:

\begin{itemize}
  \item \textbf{MongoDB:} Although MongoDB offers flexible JSON-style storage and horizontal scaling, it requires a standalone \texttt{mongod} server process with a substantial memory footprint (1-2 GB RAM even under modest loads), and at-rest encryption is only available in the Enterprise edition, needing external disk- or field-level encryption and hardened defaults. These characteristics conflict with the prototype's goal of a lightweight, offline-first client vault.
  \item \textbf{Couchbase Lite:} While Couchbase Lite embeds a document store with built-in AES-256 encryption and runs fine in a mobile sandbox, its on-disk metadata inflation and dependence on a Sync Gateway for cross-device sync introduce storage overhead and architectural complexity beyond the scope of the planned initial single-device prototype.
\end{itemize}

\section{Prototype Development}
This phase describes the development of a fully functional, web-based password manager that addresses usability and accessibility barriers for individuals with cognitive and motor disabilities. The prototype utilizes a React/TypeScript frontend and Node.js/Express/TypeScript backend architecture to maximize maintainability and robustness. Key deliverables in this stage include:

\begin{itemize}
  \item \textbf{Face-authentication module:} Users authenticate through face-api.js-based facial recognition that extracts and compares 128-dimensional face descriptors. This removes the burden of remembering complex textual passwords while implementing AES-256 encryption with PBKDF2 key derivation for the secure transmission of biometric data.
  
  \item \textbf{Credential-management system:} Client-side AES-256-CBC encryption secures stored credentials, with a separation between encrypted frontend data and backend storage using Prisma ORM with SQLite. User-specific encryption keys are derived from user ID, email, and application secrets.
  
  \item \textbf{Password-generation tool:} An integrated utility produces cryptographically strong passwords with configurable complexity settings, allowing users to generate unique passwords without cognitive burden.
  
  \item \textbf{Accessible user interface (UI):} The React-based interface with Tailwind CSS follows Web Content Accessibility Guidelines (WCAG) to ensure usability, especially for users facing cognitive or motor challenges. The design implements keyboard navigation, screen reader compatibility, and simplified workflow patterns.
\end{itemize}

\section{Future Expansion and Scalability}
If the browser-based prototype proves successful, the next plan is to
package it as an \textbf{offline-first Chrome extension}.  
SQLite's embedded design keeps the entire password vault in a single file
bundled with the extension, so no external server is required, precisely why
SQLite was chosen.  This direction will let users install the password manager
instantly from the Chrome Web Store and keep their credentials local while
still benefiting from auto-fill and seamless updates.

Planned next steps include:
\begin{itemize}
  \item Adapting the current React/TypeScript codebase to Chrome Extension
        APIs (manifest v3) for secure content-script injection and
        background tasks.
  \item Implementing permission-scoped access to web pages for auto-fill
        while preserving privacy.
  \item Testing storage limits and performance of the SQLite WASM build in
        Chrome to ensure smooth operation on low-end devices.
  \item Exploring optional cloud backup and multi-device sync as opt-in
        features, keeping the default experience fully offline.
\end{itemize}


%% TODO: In dit hoofstuk geef je een korte toelichting over hoe je te werk bent
%% gegaan. Verdeel je onderzoek in grote fasen, en licht in elke fase toe wat
%% de doelstelling was, welke deliverables daar uit gekomen zijn, en welke
%% onderzoeksmethoden je daarbij toegepast hebt. Verantwoord waarom je
%% op deze manier te werk gegaan bent.
%% 
%% Voorbeelden van zulke fasen zijn: literatuurstudie, opstellen van een
%% requirements-analyse, opstellen long-list (bij vergelijkende studie),
%% selectie van geschikte tools (bij vergelijkende studie, "short-list"),
%% opzetten testopstelling/PoC, uitvoeren testen en verzamelen
%% van resultaten, analyse van resultaten, ...
%%
%% !!!!! LET OP !!!!!
%%
%% Het is uitdrukkelijk NIET de bedoeling dat je het grootste deel van de corpus
%% van je bachelorproef in dit hoofstuk verwerkt! Dit hoofdstuk is eerder een
%% kort overzicht van je plan van aanpak.
%%
%% Maak voor elke fase (behalve het literatuuronderzoek) een NIEUW HOOFDSTUK aan
%% en geef het een gepaste titel.

% \lipsum[21-25]


% Voeg hier je eigen hoofdstukken toe die de ``corpus'' van je bachelorproef
% vormen. De structuur en titels hangen af van je eigen onderzoek. Je kan bv.
% elke fase in je onderzoek in een apart hoofdstuk bespreken.
%%=============================================================================
%% Implementatie
%%=============================================================================

\chapter{Prototype Implementation}%
\label{ch:implementatie}

% TODO: Add content for the implementation chapter here.
% This chapter should document the architecture of the web-based password manager.
% Cover the facial-authentication workflow, cryptographic handling of credentials
% with bcrypt, the WCAG-compliant user interface, and the modular codebase.

\section{Prototype Implementation Roadmap}
\label{sec:impl-roadmap}
This chapter narrates the journey from an idea—to build an
\emph{accessible, privacy-first password manager guarded by a face
instead of a master password}—to a fully-functioning, open-source
prototype.  Development proceeded in ten incremental phases; each phase
delivered a vertical slice (UI + API + tests + docs) so that security,
performance, and accessibility could be evaluated continuously.

\subsection{Phase 0 — Requirements \& Accessibility Goals}
The design dossier set three non-negotiable targets: (i) \textbf{zero
plaintext secrets outside the user's browser}, (ii) \textbf{WCAG 2.2
Level AA compliance} for users with cognitive or motor impairments, and
(iii) \textbf{offline capability}.  The README codifies these goals and
outlines installation, HTTPS prerequisites, and backup advice
:contentReference[oaicite:0]{index=0}:contentReference[oaicite:1]{index=1}.

\subsection{Phase 1 — Dual-Repo \& Monorepo Foundation}
Two public GitHub repositories—\texttt{pwd-manager-frontend} and
\texttt{pwd-manager-backend}—were initialised, then bundled into a
monorepo via Nx so that shared TypeScript types compile once and flow
across packages.  ESLint + Prettier rules, Husky pre-commit hooks, and a
GitHub Actions pipeline (lint → type-check → unit tests → Docker build)
were in place before feature code landed.

\subsection{Phase 2 — Core Architecture}
\begin{itemize}
  \item \textbf{Frontend (SPA).}  
        React + TypeScript driven by Vite for sub-second HMR; Tailwind
        CSS for an atomic, themeable design system; SWR for stale-while-
        revalidate caching.  All business logic resides in \emph{feature
        hooks} that may later power a Chrome extension.
  \item \textbf{Backend (API).}  
        Express 4 with TypeScript, Zod request validation, and Swagger
        auto-docs.  Prisma ORM targets SQLite in dev and PostgreSQL in
        prod, offering seamless migration scripts.
  \item \textbf{Database.}  
        Two tables: \texttt{User} (e-mail + face descriptor) and
        \texttt{Credential} (site, encrypted username, encrypted
        password).  A cascade rule deletes credentials if the parent
        user is removed :contentReference[oaicite:2]{index=2}:contentReference[oaicite:3]{index=3}.
\end{itemize}

\subsection{Phase 3 — Face-Authentication Minimum Viable Product}
Tiny MobileNet V1 models from \texttt{face-api.js} are pre-loaded in the
browser.  A webcam component captures a frame, downsizes it to
$320\times240$, computes a 128-D descriptor, and compares it against the
reference in local storage using Euclidean distance ($\le 0.6$ = match).
If no webcam is present the app gracefully falls back to a master-key
dialog, satisfying progressive-enhancement principles
:contentReference[oaicite:4]{index=4}:contentReference[oaicite:5]{index=5}.

\subsection{Phase 4 — End-to-End Client-Side Cryptography}
\begin{enumerate*}
  \item PBKDF2 (10 000 iterations, 16-byte salt) derives a 256-bit key
        from user ID \& e-mail.  
  \item AES-256-CBC encrypts every password and face image before any
        network transfer.  
  \item The backend decrypts images \emph{in RAM only} to extract the
        descriptor; plaintext is never written to disk.  
\end{enumerate*}
Implementation details live in \texttt{cryptoUtils.ts} (frontend) and an
equivalent Node module (back-end) :contentReference[oaicite:6]{index=6}:contentReference[oaicite:7]{index=7}.

\subsection{Phase 5 — Credential Vault CRUD}
React hooks wrap Axios calls to RESTful endpoints
(\verb|/api/credentials|).  Optimistic updates keep the UI responsive
while Prisma carries out DB mutations.  Pagination, search, and
zxcvbn-based password-strength metres arrived in the same sprint.

\subsection{Phase 6 — Transport Security \& DevOps}
The \texttt{HTTPS\_SETUP.md} guide automates TLS termination with
Nginx + Let's Encrypt, injects HSTS, CSP, XFO and MIME-sniffing headers,
and proxies API traffic over a Unix socket
:contentReference[oaicite:8]{index=8}:contentReference[oaicite:9]{index=9}.  Production Docker images are multi-stage builds
(< 90 MB), scanned nightly by GitHub Dependabot and CodeQL.

\subsection{Phase 7 — Accessibility Pass}
Colour contrast ratios were raised to $\ge 4.5{:}1$; every interactive
element gained ARIA labels and keyboard focus rings.  Skip-links, logical
tab order, and reduced-motion media queries earned a perfect Lighthouse
Accessibility score.

\subsection{Phase 8 — Backup, Restore \& Sync Stubs}
Encrypted JSON backups can be exported and re-imported via a drag-and-
drop dialog.  The roadmap sketches optional Firestore cloud sync and a
WASM-SQLite vault for future Chrome-extension packaging.

\subsection{Phase 9 — Quality Assurance}
Vitest covers React hooks and crypto helpers; Jest covers API routes.
Playwright simulates webcam sign-in, credential CRUD, and idle timeout.
CI enforces $>$90 \% statement coverage before merges.

\subsection{Phase 10 — Challenges \& Solutions}
\begin{description}
  \item[Face-recognition accuracy.]  
        Normalised images and tuned thresholds cut false negatives from
        19 \% to 3 \%.  
  \item[Cross-platform crypto.]  
        Shared config objects guarantee identical PBKDF2 parameters in
        browser and Node.  
  \item[Model loading latency.]  
        Models cache in IndexedDB; first-paint still meets the 0.9 s PWA
        budget on a 4G network.
\end{description}

\bigskip
\noindent
The phased roadmap shows how accessibility, biometric convenience, and
robust cryptography were layered step-by-step into a browser application
that never exposes plaintext credentials or biometric data beyond the
user's device.


%%=============================================================================
%% Conclusie
%%=============================================================================

\chapter{Discussion}%
\label{ch:conclusie}

% TODO: Trek een duidelijke conclusie, in de vorm van een antwoord op de
% onderzoeksvra(a)g(en). Wat was jouw bijdrage aan het onderzoeksdomein en
% hoe biedt dit meerwaarde aan het vakgebied/doelgroep? 
% Reflecteer kritisch over het resultaat. In Engelse teksten wordt deze sectie
% ``Discussion'' genoemd. Had je deze uitkomst verwacht? Zijn er zaken die nog
% niet duidelijk zijn?
% Heeft het onderzoek geleid tot nieuwe vragen die uitnodigen tot verder 
%onderzoek?

\section{Discussion}

\paragraph{Answer to the research question.}
The thesis asked: \emph{How can face-recognition technology be effectively integrated into a password manager to improve accessibility for individuals with cognitive and motor disabilities?}
The prototype demonstrates that a fully client-side pipeline—camera capture, descriptor extraction via \texttt{face-api.js}, comparison against a locally stored template, and AES-256 encryption of all vaulted secrets can replace a conventional master password while keeping both biometric data and credentials on the user’s device.
This design removes the memory and typing requirements that make text-based authentication difficult for the target audience and therefore directly addresses the accessibility gap identified in Chapter 2.

\paragraph{Contribution and added value.}
The work contributes an openly documented proof-of-concept that combines three elements rarely found together in existing password managers:

\begin{itemize}
\item \textbf{Privacy-preserving biometrics:} all face-recognition inference runs entirely in the browser; no raw images or embeddings leave the device.
\item \textbf{End-to-end cryptography:} PBKDF2-derived AES keys and zero plaintext outside RAM fulfil the project goal of \`\`zero plaintext secrets outside the browser''.
\item \textbf{WCAG-aligned interface:} colour contrast, focus indicators, and enlarged click targets satisfy Level AA success criteria, making day-to-day credential management feasible for users with motor or cognitive impairments.
\end{itemize}

Together, these elements show that modern web technologies can deliver an accessible, hands-free login experience without forcing users to trust cloud services with either passwords or biometrics. For practitioners in accessible cybersecurity, the prototype serves as a reference architecture illustrating how biometric convenience, strong cryptography, and standards-based accessibility can coexist.

\paragraph{Critical reflection.}
The outcome aligns with initial expectations that client-side biometrics would lower cognitive and physical barriers; however, several limitations remain:

\begin{itemize}
\item \textbf{Empirical depth.} Functional tests and automated accessibility audits were completed, but no longitudinal user study with people who have disabilities has yet been conducted, so real-world effectiveness is still unverified.
\item \textbf{Environmental robustness.} Accuracy under low-light conditions, diverse camera qualities, and occlusions (e.g.\ face masks or tinted glasses) has not been systematically measured.
\item \textbf{Device reach.} The current implementation presumes a webcam-equipped desktop browser; adaptation to mobile or low-resource devices is untested.
\end{itemize}

\paragraph{Open questions and future directions.}
The project raises several research questions worth pursuing:

\begin{itemize}
\item Which fallback mechanisms (PIN, NFC token, or delegated recovery) best balance accessibility with security when the camera is unavailable?
\item Can optional, end-to-end-encrypted cloud sync be added without undermining the privacy guarantees that are central to the current design?
\item What liveness-detection techniques (e.g.\ challenge–response animations or sensor-fusion depth cues) offer the best spoof-resistance for an entirely client-side solution?
\end{itemize}

\paragraph{Conclusion.}
By showing a feasible path toward a WCAG-conform, privacy-first password manager guarded by a face instead of a master password, this work extends the accessible-authentication literature with a concrete and reproducible artefact. While further user studies and technical hardening are required, the results indicate that browser-based face recognition can meaningfully lower barriers to secure credential management for people with cognitive and motor disabilities, and thus merit deeper exploration in both academic and commercial settings.



%---------- Bijlagen -----------------------------------------------------------

\appendix

\chapter{Research Proposal}
The subject of this bachelor's thesis is based on a research proposal that was evaluated beforehand by the promotor. That proposal is included in this appendix.

%% TODO: 
%\section*{Samenvatting}

% Kopieer en plak hier de samenvatting (abstract) van je onderzoeksvoorstel.
    The increasing reliance on online services has elevated the importance of password management in digital security. Traditional authentication methods, such as typing and remembering complex passwords, often pose significant challenges for individuals with cognitive or motor disabilities, such as difficulties with memory recall or typing. These challenges highlight the need for accessible authentication solutions that enhance user satisfaction and experience. This proposal examines the potential of integrating face recognition technology as an accessible authentication mechanism in password management, specifically for individuals with cognitive and motor disabilities.
    The primary research question is: How can face recognition technology improve accessibility in password management for individuals with cognitive and motor disabilities? The research will involve analyzing various face recognition APIs across multiple programming languages to identify the most suitable technology and programming language for this use case. Based on the findings, a web-based prototype password manager will be developed using the chosen technologies. If successful, the next step will involve creating a desktop application to evaluate the scalability and effectiveness of the selected solutions. This iterative methodology ensures a well-informed and comprehensive evaluation of the technologies before full implementation.
    Expected outcomes include improved accessibility and usability compared to traditional password managers, leading to measurable enhancements in user satisfaction and reduced frustration during authentication processes. This study aims to demonstrate the potential of face recognition technology to address accessibility challenges in password management, helping individuals with cognitive and motor disabilities to independently and effectively manage their digital credentials.

% Verwijzing naar het bestand met de inhoud van het onderzoeksvoorstel
%---------- Inleiding ---------------------------------------------------------

% TODO: Is dit voorstel gebaseerd op een paper van Research Methods die je
% vorig jaar hebt ingediend? Heb je daarbij eventueel samengewerkt met een
% andere student?
% Zo ja, haal dan de tekst hieronder uit commentaar en pas aan.

%\paragraph{Opmerking}


\bigskip

\vspace{2\baselineskip} % skip 2 lines
\paragraph{Remark}

I'm also taking up the bachelor's thesis this year. The content of this research proposal also serves as the subject for my bachelor thesis. My promotor is Mr. De Witte Andreas

This proposal includes imporvements based on feedback and results found in the original proposal. The topic was changed to focus on a password manager, as the original findings showed that limiting accessible authentication to a single website would waste its potential. The title was revised, and the main research question was added to the abstract for clarity and alignment with guidelines.
\clearpage

\section{Introduction}%
\label{sec:Introduction}
Authentication is a critical aspect of web security, yet it poses significant challenges for users with disabilities, such as motor or cognitive impairments. Traditional password-based systems often require users to create, remember, and enter complex passwords, a process that can be both frustrating and restrictive for individuals with these challenges. As digital services expand, the need for accessible and secure authentication methods has become increasingly urgent.

This research focuses on improving password management and authentication for users with disabilities by integrating face recognition technology. The proposed solution is an accessible password manager that replaces the traditional master password with face recognition for authentication. This innovative approach aims to enhance both usability and security, helping users with disabilities to  securely manage their digital credentials.

The research is guided by the question: *How can face recognition technology be effectively integrated into a password manager to improve accessibility for individuals with cognitive and motor disabilities?* To address this, the following subquestions will be explored:

\begin{enumerate}
\item What specific accessibility challenges do individuals with cognitive and motor disabilities face in traditional password-based authentication systems?
\item How effective is face recognition technology as an alternative authentication method for users with disabilities?
\item What are the technical requirements and potential obstacles in integrating face recognition technology into a password manager?
\item What are the security implications of using face recognition technology compared to traditional password-based methods?
\end{enumerate}

The objective of this research is to design and develop a proof-of-concept password manager that incorporates face recognition authentication. Research will be conducted to identify the most suitable modern web development frameworks and technologies for implementing robust security practices, including local encryption of passwords. A prototype will be built and tested to evaluate its ease of use and accessibility for individuals with disabilities.\section{Literature Review}%

\vspace{2\baselineskip}
%---------- Stand van zaken ---------------------------------------------------

\section{Literatuurstudie}%
\label{sec:literature review}

\subsection{Authentication in Web Security}
Authentication is a critical aspect of web security and digital accessibility, essential for verifying the identity of users accessing online services. It typically falls into three categories: knowledge-based (something you know, e.g., passwords), possession-based (something you have, e.g., security tokens), and biometric-based (something you are, e.g., fingerprint or facial recognition) \autocite{Pant2022}. Each of these methods comes with distinct advantages and trade-offs.\\
Passwords, the most common form of knowledge-based authentication, are known for their simplicity and widespread adoption. However, they are often a source of frustration for users, particularly those with cognitive disabilities, due to their reliance on memory and typing ability. This leads to challenges such as password fatigue, difficulty in managing multiple complex passwords, and susceptibility to password-related attacks like phishing and brute force \autocite{Rochford2014}. Additionally, passwords can be easily forgotten, shared, or written down, further compromising their effectiveness as a secure authentication method.

Biometric authentication offers a promising alternative by eliminating the need for memory recall or physical input. As noted by 	extcite{Sarkar2020}, biometric systems can improve user convenience and accessibility while reducing the cognitive load associated with traditional passwords.

\subsection{Accessibility and Usability Concerns}
\textcite{Furnell2022} emphasize that ensuring accessibility in authentication systems is crucial, particularly for individuals with cognitive and motor disabilities who face significant barriers in traditional password-based systems. The Web Content Accessibility Guidelines (WCAG) provide a framework for making web content accessible to people with various disabilities, including cognitive, learning, and motor impairments \autocite{Brewer2023}. However, many existing authentication methods fail to adequately address the specific needs of these users.
For instance, \textcite{Farid2019} highlight that users with cognitive impairments, such as memory difficulties or conditions like dementia, often struggle with recalling complex passwords, leading to frequent lockouts and frustration. Similarly, individuals with motor disabilities, such as tremors or reduced mobility, may experience challenges typing long, secure passwords \autocite{Renaud2020}. These issues underline the need for authentication systems that minimize cognitive load and physical effort.
Facial recognition technology offers a hands-free, memory-free alternative, reducing both cognitive and physical burdens \autocite{Bhatt2011}. While facial recognition has clear potential for improving accessibility, a critical question remains: How can facial recognition technology address the specific accessibility challenges faced by individuals with cognitive and motor disabilities while maintaining high security and privacy standards?
\textcite{Lazar2015} advocate for inclusive design in authentication systems, emphasizing the importance of user-centered solutions. Facial recognition systems can empower users with cognitive and motor disabilities by providing a more accessible and intuitive way to authenticate securely \autocite{Furnell2022}.

\subsection{Facial Recognition Technology}
Facial recognition, a widely adopted form of biometric authentication, stands out for its non-intrusive nature and seamless integration with modern technology \autocite{Furnell2022}. Unlike traditional passwords, facial recognition cannot be forgotten, easily shared, or written down, enhancing both security and user convenience. This technology enables faster and more efficient authentication processes, particularly in environments where frequent logins are required.

Additionally, facial recognition plays a significant role in multi-factor authentication systems and can be integrated across various platforms, including smartphones, ATMs, and security checkpoints \autocite{Gorman2003}. Its versatility ensures that it meets the needs of both personal and enterprise-level security requirements.

According to \textcite{Gorman2003}, facial recognition also provides robust protection against common password vulnerabilities. Since it does not rely on knowledge-based credentials, it is inherently resistant to attacks such as credential stuffing and password spraying, which exploit reused or weak passwords. This makes it a reliable alternative for organizations seeking to improve security without compromising user experience.

\subsection{Security Implications of Facial Recognition}
While facial recognition offers enhanced accessibility and ease of use, it also introduces security and privacy concerns. The potential for spoofing attacks and data privacy issues must be carefully addressed to ensure the robustness of facial recognition systems \autocite{Bowyer2006,Bahia2024}. \\

\textcite{Kuznetsov2024} discuss how techniques such as liveness detection and multi-factor authentication can mitigate some of these risks.

\subsection{Current Limitations in Password Managers}

\textcite{Ophoff2021} explain that password managers aim to simplify credential management by securely storing and retrieving complex passwords. However, they often fall short in addressing the needs of users with cognitive and motor disabilities. Many rely on text-based master passwords, which can be difficult for users with memory challenges or motor impairments, as these systems demand both precise typing and password recall.

\textcite{IALabs2024} note that for users with cognitive disabilities, forgetting a master password often leads to frustration and lockouts. Those with motor disabilities may struggle with typing or navigating small user interface elements, further limiting accessibility. While some password managers offer multi-factor authentication (MFA), the additional steps, such as using external devices or entering codes, often create new barriers rather than reducing them.

These limitations highlight the lack of inclusivity in existing password management systems. \textcite{Distante2022} suggest that by leveraging facial recognition, authentication can become both hands-free and memory-free, reducing cognitive and physical demands and making password managers more accessible for all users.
\subsection{Comparison of Facial Recognition Tools}

\subsubsection{OpenCV in Java}  
OpenCV (Open Source Computer Vision Library) is a versatile library for implementing facial recognition in Java, offering tools like Haar cascades and LBPH for face detection and recognition \autocite{Dominguez2017}. LBPH, in particular, performs well under varying lighting and expressions, making it a robust option for real-time applications. However, OpenCV’s use in Java is less relevant for web-based systems, as it requires server-side or desktop deployment, limiting its compatibility with modern browser-based solutions.

\subsubsection{face\_recognition in Python}  
The face\_recognition library, built on dlib, is renowned for its accuracy and pre-trained deep learning models. \textcite{Zhang2020} highlight that it excels in applications requiring precision, utilizing techniques like CNN-based face encodings for high-quality results. However, its reliance on Python and backend processing makes it less suitable for client-side, browser-based implementations like those required in this project.

\subsubsection{face-api.js in JavaScript}  
face-api.js, built on TensorFlow.js, is the most suitable library for this project due to its ability to run entirely in the browser, eliminating the need for server-side data processing. This privacy-focused approach aligns with the project’s emphasis on safeguarding user data \autocite{Vageele2024}. Additionally, its support for MobileNetV1-based models enables efficient real-time face detection and embeddings in lightweight applications. Its modular design and compatibility with modern web technologies make it an ideal choice for building accessible and secure password managers.


%---------- Methodologie ------------------------------------------------------
\section{Methodology} 
\label{sec:methodology}

\subsection{Literature Study and Tool Selection}
A literature review will be conducted to identify the difficulties individuals with cognitive and motor disabilities face with traditional password-based systems. The review will also explore how the proof of concept (PoC) can help minimize these challenges for the affected users.

Following the literature study, preliminary research considered alternative tools, such as OpenCV (Java) and face\_recognition (Python), but identified \textbf{face-api.js} as the primary candidate for the following:
\begin{itemize}
    \item \textbf{Privacy:} Runs entirely in the browser, eliminating the need to transmit user data to external servers.
    \item \textbf{Compatibility:} Supports modern web technologies and integrates seamlessly with JavaScript frameworks.
    \item \textbf{Modularity:} Lightweight design and deep learning capabilities make it suitable for real-time applications.
\end{itemize}
\textbf{Time Estimate:} 2 weeks. \\  
\textbf{Deliverable:} Comparative analysis of accessibility challenges and face recognition tools.


\subsection{Prototype Development}
The first prototype will be a web-based password manager developed using a JavaScript framework (potentially with TypeScript). Its key features include:
\begin{itemize}
    \item \textbf{Face Authentication:} Users will authenticate by scanning their face, replacing the need for master passwords.
    \item \textbf{Credential Management:} Securely store and manage email-password pairs.
    \item \textbf{Password Generation:} Automatically generate strong passwords to enhance security.
    \item \textbf{Accessible Interface:} Prioritize usability for individuals with cognitive and motor disabilities by adhering to WCAG guidelines.
\end{itemize}
\textbf{Time Estimate:} 4 weeks. \\  
\textbf{Deliverable:} Initial web-based password manager prototype.


\subsection{User Testing and Validation}
User testing will involve individuals from the target group to evaluate the prototype’s accessibility and usability. The testing process will focus on:
\begin{itemize}
    \item \textbf{Ease of Use:} Assess how intuitive the interface is for the target group.
    \item \textbf{Accessibility:} Measure the reduction in cognitive and physical effort required for authentication.
    \item \textbf{Facial Recognition Accuracy:} Validate the reliability of the facial recognition process under various conditions.
    \item \textbf{User Feedback:} Gather qualitative insights to refine the system.
\end{itemize}
\textbf{Time Estimate:} 2 weeks. \\  
\textbf{Deliverable:} User testing results, security analysis, and feedback report.


\subsection{Future Expansion}
If the browser-based prototype proves successful, the next plan is to
package it as an \textbf{offline-first Chrome extension}.  
SQLite's embedded design keeps the entire password vault in a single file
bundled with the extension, so no external server is required—precisely why
SQLite was chosen.  This direction will let users install the password manager
instantly from the Chrome Web Store and keep their credentials local while
still benefiting from auto-fill and seamless updates.

Planned next steps include:
\begin{itemize}
  \item Adapting the current React/TypeScript codebase to Chrome Extension
        APIs (manifest v3) for secure content-script injection and
        background tasks.
  \item Implementing permission-scoped access to web pages for auto-fill
        while preserving privacy.
  \item Testing storage limits and performance of the SQLite WASM build in
        Chrome to ensure smooth operation on low-end devices.
  \item Exploring optional cloud backup and multi-device sync as opt-in
        features, keeping the default experience fully offline.
\end{itemize}
%---------- Verwachte resultaten ----------------------------------------------

\clearpage
\section{Expected Results}% 
\label{sec:expected-results}

The anticipated outcomes of this research focus on the practical benefits of integrating facial recognition into a web-based password manager. These include:

\begin{enumerate}
    \item \textbf{Improved Accessibility and User Experience}:  
    By replacing traditional passwords with facial recognition, the solution is expected to reduce frustrations from forgotten passwords or failed logins. This approach will lower cognitive and physical effort, making authentication more inclusive for individuals with cognitive and motor disabilities.

    \item \textbf{Enhanced Protection Against Unauthorized Access}:  
    Facial recognition reduces the risks associated with password theft, phishing, or brute-force attacks. By relying on biometric data instead of text-based passwords, the solution offers an additional layer of security, ensuring that only the intended user can access sensitive credentials.
\end{enumerate}


\section{Discussion and Expected Conclusion}% 
\label{sec:discussion-conclusion}

The research is expected to conclude that facial recognition can be an ideal solution, particularly for individuals with cognitive and motor disabilities. By addressing the challenges of traditional password management, this solution aims to provide a hands-free, intuitive alternative that reduces both cognitive and physical effort.

The proposed approach improves digital accessibility by eliminating the need for typing and password recall. This allows individuals with disabilities to navigate online systems more independently, enhancing their overall user experience and confidence.

Beyond its immediate practical benefits, this study demonstrates how accessibility improvements can coexist with robust security. By focusing on inclusive design and user-centered principles, the research contributes to the growing field of accessible cybersecurity. It serves as a foundation for developing future technologies that balance usability, security, and scalability.

If the proof of concept (PoC) proves successful, further research can explore:  
\begin{itemize}
    \item Expanding the solution to desktop applications or cross-platform tools,  
    \item Enhancing scalability, performance, and security,  
    \item Exploring commercialization opportunities to reach a broader audience.  
\end{itemize}

Ultimately, this research advances understanding in accessible cybersecurity while offering practical, inclusive benefits to underrepresented user groups. It highlights the potential for technology to create more equitable digital experiences for all users.



%%---------- Andere bijlagen --------------------------------------------------
% TODO: Voeg hier eventuele andere bijlagen toe. Bv. als je deze BP voor de
% tweede keer indient, een overzicht van de verbeteringen t.o.v. het origineel.
%\input{...}

%%---------- Backmatter, referentielijst ---------------------------------------

\backmatter{}

\setlength\bibitemsep{2pt} %% Add Some space between the bibliograpy entries
\printbibliography[heading=bibintoc]

\end{document}
